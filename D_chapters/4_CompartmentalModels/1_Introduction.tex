\section{Introduction}

To be able to compare our experimental DARPin-F10 data and our dose-response predictions we sought out existing influenza infection models. Non-structured \textit{in vitro} influenza kinetic models were considered a good fit for our goals, as they mostly rely on directly observed experimental quantities such as viral growth.

In this chapter we provide an overview on \textit{in vitro} influenza kinetic modelling literature, use literature data \cite{rudiger2019multiscale, schulze2009infection} to analyze functional dependencies between model parameters, and to fit a simple influenza infection model, which we then attempt to use in conjunction with the predicted DARPin-F10 dose response.

\subsection{Influenza infection modelling}

Seasonal and zoonotic influenza is a popular subject of viral modelling, which has a variety of approaches. Majority of viral models use systems of ordinary differential equations (ODE), but partial (PDE) and delay differential equations (DDE) have also been implemented.

Influenza infection dynamic models are focused primarily capturing the transmission between hosts, with the goal of informing public health decisions an assist in pandemic planning \cite{ferguson2006strategies, mcvernon2007model}.

With the advancement of the social media, a new type of influenza forecasting models has emerged \cite{pawelek2014modeling, santillana2015combining}, relying on publicly available self-reporting by users.

Structured models which include individual processes in virus replication \cite{sidorenko2004structured}, endosomal escape \cite{lagache2012modeling} and defective viral particle propagation \cite{rudiger2019multiscale} have been proposed. Their phenomenological nature means that they often rely on unobserved quantities and variables, and often don't allow for inference on specific molecular targets for intervention.

Non-structured influenza kinetic models aim to understand and quantify severity, duration, and overall progression of the infection within a host or a cell culture. \textit{In vitro} models usually have a simple and understandable structure, and rely on directly observed experimental quantities. \textit{In vivo} kinetic models usually build up on that basic structure and attempt to incorporate innate \cite{beauchemin2008modeling} and adaptive \cite{belz2002compromised} immune response. Here we primarily focus on \textit{in vitro} models, as our DARPin-F10 experimental data is coming from cell culture experiments.

Chronic infection kinetic model, originally proposed for human immunodeficiency virus (HIV) \cite{perelson2002modelling}, includes three states: $T$ - target cells, $I$ - infected cells, $V$ - viral particles, which are described as follows:

\begin{equation}
\begin{array}{rcl}
\frac{dT}{dt} &=& s T - d T - \beta T V \\
\frac{dI}{dt} &=& \beta T V - \delta I \\
\frac{dV}{dt} &=& p I - c V
\end{array}
\end{equation}

where $s$ is target cells regeneration, $d$ is natural target cells death, $\beta$ is target cells infection rate by viral particles, $\delta$ is death rate of infected cells, $p$ is viral particle production rate by infected cells, and $c$ is clearance rate of viral particles.

It's often assumed that during acute ($s = d = 0$ \cite{baccam2006kinetics}) influenza infection target cells $T$ are limited and are depleted over the course of the infection:

\begin{equation}
\begin{array}{rcl}
\frac{dT}{dt} &=& - \beta T V \\
\frac{dI}{dt} &=& \beta T V - \delta I \\
\frac{dV}{dt} &=& p I - c V
\end{array}
\end{equation}

Another commonly made assumption is that during influenza infection viral production is delayed, which is accomplished through presence of a latent eclipse phase infected cells $I_1$ and productively infected cells $I_2$ (\cite{baccam2006kinetics}):

\begin{equation}
\begin{array}{rcl}
\frac{dT}{dt} &=& - \beta T V \\
\frac{dI_1}{dt} &=& \beta T V - k I_1 \\
\frac{dI_2}{dt} &=& k I_1 - \delta I_2 \\
\frac{dV}{dt} &=& p I_2 - c V
\end{array}
\end{equation}

where $k$ is a rate of $I_1$ maturation into $I_2$.

Alternative approach to modeling this latency is through introducing a fixed delay $\tau$:

\begin{equation}
\begin{array}{rcl}
&\frac{dT}{dt} = - \beta T(t) V(t) \\
&\frac{dI}{dt} = \beta T(t-\tau) V(t-\tau) - \delta I(t) \\
&\frac{dV}{dt} = p I(t) - c V(t)
\end{array}
\end{equation}

Here, "eclipse phase" the infected cell doesn't contribute to systems dynamics. Delay model disregards variability of transition time from eclipse phase, but allows to avoid unrealistically small or large transition times from eclipse to productive phase \cite{beauchemin2008modeling}.

\cite{mohler2005mathematical,schulze2009infection} model viral production on microcarriers and introduce delay in viral production term $p I(t - \tau)$ instead.

Several models have been suggested to describe influenza infection in presence of antiviral drugs. Amantadine drugs are commonly modelled as follows:

\begin{equation}
\begin{array}{rcl}
\frac{dT}{dt} &=& - (1-\epsilon_{drug})\beta T V \\
\frac{dI}{dt} &=& \beta T V - \delta I \\
\frac{dV}{dt} &=& p I - c V
\end{array}
\end{equation}

where $\epsilon_{drug}$ is a drug efficacy, determined through dose-response $E_{max}$ model:

\begin{equation}
\epsilon_{drug} = \epsilon_{max}\frac{1}{1 + (\frac{[D]}{IC_{50}})^{-n}}
\end{equation}

where $\epsilon_{max}$ is a maximal drug effect such that $0 < \epsilon_{max} \le 1$, and $n \ge 0$ is a Hill coefficient.

\cite{beauchemin2008modeling} also examine the possibility that amantadine lengthens eclipse phase instead of inhibiting the infection:

\begin{equation}
\begin{array}{rcl}
\frac{dT}{dt} &=& - \beta T V \\
\frac{dI_1}{dt} &=& \beta T V - (1-\epsilon_{drug}) k I_1 \\
\frac{dI_2}{dt} &=& (1-\epsilon_{drug}) k I_1 - \delta I_2 \\
\frac{dV}{dt} &=& p I_2 - c V
\end{array}
\end{equation}

Similarly, neuraminidase inhibitors are often assumed to influence viral production rate $(1-\epsilon_{drug}) p I$ instead.


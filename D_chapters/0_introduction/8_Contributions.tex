\section{Contributions of this thesis}

The goal of this thesis was to demonstrate the viability of a layered multiscale modelling approach for analysis of a complex biological system. Specifically, the biological system of choice is HDAC6-mediated influenza uncoating and consequent viral infection, in normal and perturbed conditions.

In Chapter \ref{ch:TugOfWar} we use known physical and geometrical characteristics of proteins of interest, to develop a biophysical model of HDAC6-mediated influenza uncoating using two-dimensional molecular motor tug-of-war to predict capsid breakage efficiency based on the molecular motor numbers.

In Chapter \ref{ch:ReactionModels} we create a chemical master equation type models of HDAC6 complex formation. Using database data for average protein abundances within cells and literature data for reaction rates estimates as a starting point, we sample reaction rates to determine viable combinations of parameters leading to successful molecular motor recruitment by HDAC6 and consequent influenza capsid breakage. We show that resulting models are able to robustly predict experimentally observed influenza uncoating.

In Chapter \ref{ch:DARPin} we expand our HDAC6 complex formation models to include DARPin-F10 - a novel drug like protein inhibiting Ub binding by HDAC6-ZnF. Using these models we estimate DARPin-F10 effective concentrations, and corresponding dose response of influenza capsid breakage. By building up on the existing kinetic models of influenza infection, we model influenza infection outcomes in presence of DARPin-F10.
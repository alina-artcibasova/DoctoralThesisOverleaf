\section{Mathematical models allow functional analysis of biological systems}

A variety of mechanistic mathematical modelling approaches have been employed previously to analyze and elucidate the function of cellular systems and viral infections.

For example, a process through which molecular motors may exert force onto the influenza virus capsid layer is called tug-of-war. Tug-of-war involves velocity and force balance between all involved parties. Previously it has been used to describe two types of microtubule motors to model endosomal transport \cite{muller2008tug}, and to prove that competitive and cooperative modes of molecular motor transport are not mutually exclusive \cite{muller2008tug}.

A stochastic model of bidirectional microtubular transport \cite{gazzola2009stochastic} is used to describe transport of adenovirus, and allows to predict the number of recruited molecular motors and available motor binding sites on the virus.

In another example, a biophysical model with  Langevin-type approximations  describes conformational changes of endosomolytic proteins in nonenveloped viruses \cite{lagache2012modeling}. This model is used to predict escape time based on the size of the endosome and the number of viral particles inside of the endosome.

A multiscale structured model of influenza infection \cite{heldt2013multiscale} predicts that drug interventions primarily targeting viral entry lead to a simple delay of infection, while interventions targeting viral protein or vRNP  production lead to a reduction in viral titer.

These examples highlight that mathematical models coupled with existing knowledge of biological systems allows advanced predictions about underlying processes in these systems.
\section{Existing approaches in influenza computational modelling}

Seasonal and zoonotic influenza is a popular subject of viral modelling, which has a variety of approaches. The majority of viral models use systems of ordinary differential equations (ODE), but partial (PDE) and delay differential equations (DDE) have also been implemented.

The knowledge of amino acid sequence and protein folding allow to predict the structure of the viral proteins, and how they can bind to their targets or inhibitors.

On intracellular level, stochastic and structured kinetic approaches are used to model specific processes during the infection. Structured models which include individual processes in virus replication \cite{sidorenko2004structured}, endosomal escape \cite{lagache2012modeling} and defective viral particle propagation \cite{rudiger2019multiscale} have been proposed. Their phenomenological nature means that they often rely on unobserved quantities and variables, and do not allow for inference on specific molecular targets for intervention.

To describe the transmission within host – cell culture or an individual – people use non structured kinetic models. These models aim to understand and quantify the progression of the infection, and determine its resulting severity and duration \cite{beauchemin2008modeling}.

Finally, to describe the transmission between the hosts, people use agent-based or dynamic models. Influenza infection dynamic models are focused primarily capturing the transmission between hosts, with the goal of informing public health decisions and assist in pandemic planning \cite{ferguson2006strategies, mcvernon2007model}. With the advancement of social media, a new type of influenza forecasting models has emerged \cite{pawelek2014modeling, santillana2015combining, levy2018modeling}, relying on publicly available self-reporting by users.
\section{Influenza Virus Burden}

In 2004 acute lower respiratory infections (ALRI) were second leading cause of disease (429.2 million cases), and third leading cause of death worldwide \cite{world2008global} (4.2 million deaths). ALRI can be caused by a variety of pathogens, such as influenza virus, rhinovirus, coronavirus, respiratory syncytial virus and others. Previously, influenza virus was reported as the second most common pathogen identified in children with ALRI \cite{nair2011global}.

Worldwide, influenza virus infections result in large direct healthcare costs, indirect loss of productivity costs \cite{de2015systematic}, and heavy disease burden, projected to cost \$87.1 billion \cite{molinari2007annual}. World Health Organization (WHO) estimates that influenza virus leads to 3-5 million cases of severe illness, and about 290,000-650,000 respiratory deaths annually \cite{influenza_seasonal_2018}.

Influenza infection poses greater risk for pregnant women, young children, the elderly, and individuals with chronic and immunosuppressive medical conditions. Due to workplace exposure, health care workers are at high risk for getting ill and further transmitting influenza to vulnerable individuals \cite{influenza_seasonal_2018}.
\section{Discussion}

\subsection{A physically realistic tug-of-war model for IAV uncoating}

Uncoating is arguably an enigmatic step of virus entry, with only few detailed studies available for a limited set of viruses \cite{helenius2018virus, walsh2019exploitation}. As a transient step, uncoating is hard to measure quantitatively; for example, only recently have dynamic measurements in cells with complete influenza A viruses become possible using monoclonal antibodies \cite{banerjee2014influenza} or quantum dots \cite{qin2019real}. In addition, uncoating involves diverse (host and virus) components and interrelated (biophysical, biochemical, and cell biological) processes, making it difficult to integrate the relevant data and knowledge. These aspects may have contributed to limited conclusive evidence for a tug-of-war mechanism that has been proposed for several viruses \cite{banerjee2014influenza, lukic2014hiv, radtke2010plus, strunze2011kinesin}. However, the combination of biochemical analysis (Chapter \ref{ch:TugOfWar}) and mechanistic mathematical modeling (Chapter \ref{ch:ReactionModels}) described here for IAV uncoating demonstrates the feasibility of such integration, up to consistent predictions of differential infectivity of IAV strains with a single amino acid substitution in the M1 capsid protein.

Intriguingly, the model neither relies on exact knowledge of component concentrations and interaction parameters, nor on their direct estimation from experimental data except for M1 binding mutants, which could prevent truly independent predictions. We represented uncertainties explicitly, allowing us to integrate in a consistent manner relatively precise knowledge such as molecular motor forces with much less certain aspects such as binding affinities to make the model physically realistic. Despite these uncertainties being reflected in model predictions, the qualitative and often quantitative agreement with the experimental data indicates substantial robustness to assumptions made in model construction. In general, we aimed to identify core mechanisms that suffice to explain the collected experimental observations on IAV uncoating using a minimal model. We assume, for example, singular bonds between M1 proteins in the capsid, no influence of endosomal transport on the tug-of-war mechanisms, and maximal capsid breakage for reaction rate optimization. This minimal nature of the model also implies that pleiotropic effects of perturbations are not well captured. For example, discrepancies for nocadazole experiments may arise because the drug not only disrupts microtubule dynamics, but also affects cell metabolism and viral traffic \cite{naghavi2017microtubule}.

\subsection{Mechanisms of IAV uncoating}

For a mechanistic interpretation, it is important that our model does not represent directional transport by multiple motors associated to a cargo \cite{hancock2014bidirectional}, but rather the generation of mechanical forces by the molecular motors. The high consistency between our model predictions and experimental data, however, argues against a mere positioning effect of the capsid inside the cell as a condition for uncoating, as hypothesized in early tug-of-war concepts \cite{lukic2014hiv, radtke2010plus}. The recent discovery of two host proteins activated early in infection and facilitating IAV uncoating supports this notion. Epidermal growth factor receptor pathway substrate 8 (EPS8) possibly fulfills a function in uncoating by modulating actin dynamics \cite{larson2019eps8}. G protein-coupled receptor kinase 2 (GRK2) appears to act via non-canonical targets \cite{yanguez2018phosphoproteomic}, and it is known to control cytoskeletal dynamics as well.

In more detail, a key element of our proposed mechanism is the ubiquitin-mediated binding between HDAC6 and molecular motors. Future studies should identify the source and nature of the binding-mediating ubiquitin chains. They could be provided by the virus \cite{banerjee2014influenza}, or by local or general host sources. Our models, however, suggest that viral ubiquitin alone does not support efficient capsid breakage (Figure \ref{figure:ReactionModelSchemes}). With the current experimental data, we cannot conclusively select between the "Symmetric" or "Asymmetric" model variants. The "Symmetric" variant is less predictive for most of the endosomal uncoating experiments (Figure \ref{figure:reactionModelPredictions}) and the "Asymmetric" variant seems more realistic based on our collaborators' observation that HDAC6 ZnF mutant enhances HDAC6-dynein interaction and reduces HDAC6-myosin-10 interaction. Discrepancies between models and experiments may arise due to involvement of other factors, or other types of polyubiquitin in the HDAC6-dynein interaction that are not accounted for. These discrepancies may also indicate the areas where we lack an understanding of underlying mechanisms. The "Symmetric" model variant assumes that binding of myosin and dynein both are conditional on ubiquitin, in contrast to the "Asymmetric" variant, where dynein binding is independent of ubiquitin (Figure \ref{figure:ReactionModelSchemes}). Experimental perturbations to ubiquitin, myosin, and dynein recruitment, and their combinations, compared to corresponding model perturbations, could help to conclusively identify the true pattern of ubiquitin-dependent motor recruitment.

\subsection{Implications for other viruses and antiviral treatment}

An intriguing open question is if, and to what extent the tug-of-war uncoating mechanism transfers to other viruses. Adenovirus uncoating may have the closest similarity, with the nuclear pore complex (NPC) acting as a static ‘hold’ that facilitates disruptive force generation by microtubule-associated kinesins (where the NPC-equivalent are dyneins for IAV) \cite{flatt2019adenovirus,greber2019adenovirus}. For HIV-1 the picture is more complex: initial opening of the capsid is the rate-limiting step for uncoating \cite{marquez2018kinetics}, but it is unclear if \textit{in vivo} the initial breach is caused by a tug-of-war mechanism \cite{rawle2018toward}, pressure increase in the capsid due to reverse transcription \cite{rankovic2017reverse}, or a dedicated uncoating receptor akin to the case for enteroviruses \cite{zhao2019human}. Intriguingly, the proposed uncoating receptor for HIV-1 is the $\beta$-karyopherin transportin-1 (TRN-1/TNPO1) \cite{fernandez2019transportin}, the same host protein that induces the debundling of IAV ribonucleoproteins after capsid breakage \cite{miyake2019influenza, yamauchi2020influenza}. Differences between HIV-1 and IAV uncoating may simply result from, for example, transcription initiation of viral genes in the cytoplasm and nucleus, respectively. We argue that such parallels nevertheless warrant more detailed experimental \textit{in vivo} probing of potential tug-of-war mechanisms in HIV-1 uncoating, preferably integrated with quantitative mechanistic modeling as shown here.

Finally, because host components are particularly attractive targets for novel antiviral treatment strategies, and because the HDAC6 / aggresome pathway appears to be used by multiple viruses (manuscript in preparation), we see potential for antiviral treatment development by targeting HDAC6-mediated uncoating. Compared to recently proposed targets such as EPS8 \cite{larson2019eps8} and GRK2 \cite{yanguez2018phosphoproteomic} such developments appear promising for two main reasons: (i) a clear mechanistic picture relating several identified host components to viral infectivity via uncoating, and (ii) a predictive model that allows one to account for differences between viral strains. In addition to an extended experimental analysis discussed above, refined mathematical models could, for example, include intracellular transport or structure-function relationships for capsid and individual proteins. The potential transfer of the proposed mechanisms and of the overall experimental-computational framework beyond influenza, in addition, could make such efforts worthwhile.

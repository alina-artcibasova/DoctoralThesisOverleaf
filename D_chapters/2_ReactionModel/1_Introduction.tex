\section{Introduction}

Using a biophysical model of capsid disassembly (Chapter \ref{ch:TugOfWar}) we were able to show that a Tug-of-War mechanism of uncoating is feasible for certain combinations of molecular motors attached to the viral fusion pore. However, we do not know whether such motor recruitment is feasible \textit{in vivo}.

More generally, in part due to the difficulties of experimental measurements, there is no consistent understanding or quantitative description for any virus of how the various virus-host interactions integrate with the viral capsid mechanics to impact uncoating and ultimately infectivity. Here, we therefore combine computational modeling with experimental analysis insights to quantitatively elucidate the mechanisms of HDAC6-mediated uncoating of influenza A virus \textit{in vivo}. 

Previous studies highlighted the role of ubiquitin (Ub) in influenza uncoating \cite{rudnicka2016ubiquitin}: influenza virus carries high numbers of cellular Ub B \cite{hutchinson2014conserved}, itchy E3 ubiquitin protein ligase (ITCH) ubiquitinates M1 capsid protein, promoting influenza escape from late endosomes \cite{su2013pooled}, depletion of E3 ubiquitin ligase cullin 3 inhibited influenza ucoating \cite{hubner2012cullin, huotari2012cullin}. These results, coupled with the crucial role of HDAC6-ZnF during influenza uncoating \cite{banerjee2014influenza} suggest that Ub may serve as an interface between capsid protein M1 and HDAC6.

However, biochemical analysis by our collaborators from Patrick Matthias' group at FMI identified an essential role for unanchored ubiquitin chains in bridging HDAC6-myosin interaction. The sizes and geometries of HDAC6 and molecular motors render these two possibilities for HDAC6-Ub interactions mutually exclusive. Furthermore, viral Ub interfacing HDAC6-M1 binding would lead to competition between viral and cellular Ub. On the other hand, if Ub is involved in molecular motor recruitment instead, cellular Ub would assist the binding, and viral Ub would simply support it via co-localization.

Using those experimental insights we developed three detailed biochemical-biophysical model that use capsid breakage probabilities, determined by our Mass-Spring model (Chapter \ref{ch:TugOfWar}). We showed that competitive model with viral Ub interfacing HDAC-M1 interaction was unable to recruit molecular motors for any combinations of parameters. Meanwhile, models with Ub assisting motor recruitment robustly predict IAV uncoating efficiency \textit{in vivo} in unperturbed infection as well as across various perturbed conditions.

Our collaborators from Mirco Schmolke's group at University of Geneva demonstrated that the infectivity difference between two clinical influenza strains (H1N1 and H3N2) depends on a single amino acid variation between their M1 proteins. We use our combined biochemical-biophysical models to estimate how this difference affects binding to HDAC6, HDAC6-dependent uncoating, and infectivity. The results are consistent between experiments and model predictions, further supporting our tug-of-war mechanism for influenza A uncoating.
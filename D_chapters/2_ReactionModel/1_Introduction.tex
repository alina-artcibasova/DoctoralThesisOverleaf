\section{Introduction}

Using biophysical model of capsid disassembly (Chapter \ref{ch:TugOfWar}) we were able to show that Tug-of-War mechanism of uncoating is feasible for certain combinations of molecular motors attached to the viral fusion pore. However, we don't know whether such motor recruitment is feasible \textit{in vivo}.

More generally, in part due to the difficulties of experimental measurements, there is no consistent understanding or quantitative description for any virus of how the various virus-host interactions integrate with the viral capsid mechanics to impact uncoating and ultimately infectivity. Here, we therefore combine computational modeling with experimental analysis insights to quantitatively elucidate the mechanisms of HDAC6-mediated uncoating of influenza A virus \textit{in vivo}. 

Biochemical analysis by our collaborators from Patrick Matthias' group at FMI identified an essential role for unanchored ubiquitin chains in bridging HDAC6-myosin interaction. Using those experimental insights we developed a detailed biochemical-biophysical model that uses capsid breakage probabilities, determined by Mass-Spring model (Chapter \ref{ch:TugOfWar}) to robustly predict IAV uncoating efficiency \textit{in vivo}, in unperturbed infection as well as across various perturbed conditions.

Our collaborators from Mirco Schmolke's group at University of Geneva demonstrated that the infectivity difference between two clinical influenza strains (H1N1 and H3N2) depends on a single amino acid variation between their M1 proteins. We use our combined biochemical-biophysical model to estimate how this difference affects binding to HDAC6, HDAC6-dependent uncoating, and infectivity. The results are consistent between experiments and model predictions, further supporting our tug-of-war mechanism for influenza A uncoating.
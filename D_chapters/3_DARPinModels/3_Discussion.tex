\section{Discussion}

When dose-response models are discussed, usually they are based on experimentally measured effects in, for example, cell culture. As far as we are aware, the use of another biochemical-biophysical model for such a purpose is a conceptual novelty. Admittedly, the underlying biochemical model relies on simple bimolecular reactions equations, which might not necessarily be the case for actual proteins involved. However, the overall approach is not at all limited to bimolecular reactions, and can be extended to include any amount of interaction detail, given reasonable estimates of concentrations and reaction rates.

Our estimates of DARPin-F10 effective concentrations rely on experimental observations of DARPin-F10 efficacy. Under the assumption that DARPin-F10 inhibits HDAC6-mediated influenza uncoating, the uncoating experiment (Figure \ref{figure:darpinUncoatingExperimental}) should provide the best estimate. However, in this uncoating experiment a much higher viral titer with MOI = 30 PFU/ml has been used (in comparison, \cite{banerjee2014influenza} report MOI = 1 in MEF experiments), possibly leading to lower estimates of DARPin-F10 efficiency. Furthermore, our model predictions rely on knowledge of reaction rates. While experimental data provide an estimate of dissociation rate $K_{HF}$ between DARPin-F10 and HDAC6, we use an average rate of bimolecular reactions \cite{bionumbersbimolrate} as an estimate for on-rate $k_{HF}$. Given that it controls how quickly the binding happens, it is possible we underestimate the actual amount of HDAC6-bound DARPin-F10. This may lead to a higher estimate for effective concentrations than what is actually required for uncoating inhibition.

Non-structured kinetic models are quite attractive in their basic idea - by simply determining the initial amounts of cells and the virus we can predict the trajectories of the viral infection. However, quite often their use is limited to simply calculating the reproductive number $R_0$. One reason for such limited use is implicit assumptions coming from fitted parameters. This aspect is obvious when we examine the variability of model parameters between different publications \cite{smith2011influenza}. Another difficulty is differences in preparation protocols, as we can clearly see in infectious fraction differences between \cite{rudiger2019multiscale} and \cite{schulze2009infection} (Figure \ref{figure:infectiousFraction}). Finally, the use of a  non-structured model makes it difficult to make any mechanistic predictions which are not already apparent from the underlying data.

Here we tried to circumvent this issue by combining those non-structured kinetic models with simple data-driven models, which aim to untangle the implicit relationships between initial conditions and parameters (Figures \ref{figure:delayMoiValidation}, \ref{figure:infectiousFraction}). Unfortunately, infectious fraction analysis was complicated by strong differences between the two used datasets, and use of just a single dataset with only three datapoints is hardly justified for this purpose. Nevertheless, we believe that this approach warrants future attempts.

Our suggested relationship between $\tau_{delay}$ and MOI is only based on five datapoints. Despite this sparsity of data, it seems to capture the results in human and cell culture reasonably well (Figure \ref{table:delayTauValidation}). However, within the framework of kinetic models we were not able to conclusively prove an advantage of using it over a simple T$_Hill$IRVV$_i$, $\tau_{delay} = const$ fit. The two models seem to describe the fitting and validation data similarly well, and the only improvement our T$_Hill$IRVV$_i$, $\tau_{delay} = f(MOI)$ offers is a reduction in the number of fitted kinetic model parameters by one. Here we assume the delay value $\tau_{delay}$ can be directly observed from the viral growth trajectories. Judging from the simulation results of $\tau_{delay} = const$ that might not necessarily be the case, and it is entirely possible that MOI rather affects, for example, infection rate $\beta$ (since higher number of viral particles would increase the probability of target cells getting infected). 

Given these challenges with implementing a basic robust kinetic infection model, our conclusions about the role of DARPin-F10 have to be taken with a grain of salt. Specifically, our measure of DARPin-F10 efficacy on viral growth curves is based on the total produced virus trajectories. Our fitting and validation of kinetic models highlighted that even our best kinetic models struggle to fully capture viral production. Specifically, amantadine- and neuraminidase inhibitor-like model variants failed to capture the gradual decrease in viral production observed late in the experiment (Figures \ref{figure:amantadineLikeF210}, \ref{figure:neuraminidaseInhibitorLikeF62}, > 48 h). As previously discussed, during the experiment the infectious fraction decreases over time (Figure \ref{figure:infectiousFraction}), possibly due to accumulation of DIPs, but in our simulations we assume that infectious fraction stays constant. Ultimately, it may be useful to not only focus the experiments on total viral production, but also to include cellular trajectories and infectious viral production, since uncoating inhibition may increase the production of DIPs, reducing the observed infectious fraction.

Despite these difficulties, based on our simple kinetic models of DARPin-F10 effect we can confirm the observation \cite{heldt2013multiscale} that drug interventions targeting viral production are more efficient at achieving reduction in viral titer than interventions primarily targeting viral entry. Intriguingly, the neuraminidase inhibitor-like model captures the kinetics of drug-like intervention in uncoating inhibition better than the amantadine-like model. It is unclear why uncoating inhibition would have disruptive effect on viral production, but inhibition of HDAC6-Ub binding may have further downstream effects. It has been previously reported that HDAC6 negatively regulates influenza infection by deacetylating PA subunit of influenza RNA polymerase \cite{chen2019hdac6}. On the other hand, HDAC6 has been shown to assist viral components trafficking through microtubule deacetylation \cite{husain2014histone}. Thus, along with inhibiting HDAC6-mediated uncoating DARPin-F10 presence may further restrict viral production through other HDAC6-mediated processes.
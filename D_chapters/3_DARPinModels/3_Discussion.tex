\section{Discussion}

Usually, when dose-response models are discussed, the underlying assumption is that they are based on experimentally measured effects in, for example, cell culture. As far as we're aware, the use of another biochemical-biophysical model for such purpose provides a conceptual novelty. Admittedly, the underlying biochemical model relies on simple bimolecular reactions equations, which might not necessarily be the case for actual proteins involved. However, the overall approach is not at all limited to bimolecular reactions, and can be extended to include any amount of interaction detail, given reasonable estimates of concentrations and reaction rates.

The resulting estimates of effective concentrations we have obtained rely on experimental observations of DARPin-F10 efficacy. Given the expected direct effect on HDAC6-mediated uncoating the uncoating experiment (Figure \ref{figure:darpinUncoatingExperimental}) should provide the best estimate. However, in this uncoating experiment a much higher viral titer with MOI = 30 PFU/ml has been used (in comparison, \cite{banerjee2014influenza} report MOI = 1 in MEF experiments), possibly leading to lower estimates of DARPin-F10 efficiency. Further, our model predictions rely on knowledge of reaction rates. While experimental data provides an estimate of dissociation rate $K_{HF}$ between DARPin-F10 and HDAC6, we use an average rate of bimolecular reaction \cite{bionumbersbimolrate} as an estimate for on-rate $k_{HF}$. Given that it controls how quickly the binding happens, it is possible we underestimate the actual amount of HDAC6-bound DARPin-F10, which leads to higher estimate for effective concentrations than what's actually required for uncoating inhibition.

Non-structured kinetic models are quite attractive in their basic idea - by simply determining the initial amounts of cells and the virus we can predict the trajectories of the viral infection. However, quite often their use is limited to simply calculating reproductive number $R_0$, and further application is complicated by implicit assumptions coming from fitted parameters, apparent in the variability of parameters between different publications \cite{smith2011influenza}, and specifics of preparation protocols, as we can clearly see in infectious fraction differences between \cite{rudiger2019multiscale} and \cite{schulze2009infection} (Figure \ref{figure:infectiousFraction}). As well as, use of non-structured model makes it difficult to make any mechanistic predictions which aren't already apparent from the underlying data.

Here we tried to circumvent this issue by combining those non-structured kinetic models with data-driven models untangling the implicit relationships between initial conditions and parameters (Figures \ref{figure:delayMoiValidation}, \ref{figure:infectiousFraction}). Unfortunately, infectious fraction analysis was complicated by strong differences between the two used datasets, and use of just single dataset with only 3 datapoints is hardly justified for such purpose. Nevertheless, we believe that it could prove to be a useful endeavor in the future.

The relationship between $\tau_{delay}$ and MOI that we suggest, while only based on 5 datapoints seems to capture the results in human and cell culture reasonably well (Figure \ref{table:delayTauValidation}). However, within the framework of kinetic models we weren't able to conclusively prove an advantage of using it over a simple T$_Hill$IRVV$_i$, $\tau_{delay} = const$ fit. The two models seem to describe the fitting and validation data similarly well, and the only improvement our T$_Hill$IRVV$_i$, $\tau_{delay} = f(MOI)$ offers is a reduction in number of fitted kinetic model parameters by 1. One possibility to consider, is that we assume the delay value $\tau_{delay}$ can be directly observed from the viral growth trajectories. Judging from the simulation results of $\tau_{delay} = const$ that might not necessarily be the case, and it's entirely possible that MOI rather affects, for example, infection rate $\beta$, which would make sense given that higher number of viral particles would increase the probability of target cells getting infected. 

Given these difficulties with implementing a basic robust kinetic infection model, our conclusions about the role of DARPin-F10 have to be taken with a grain of salt. Specifically, our measure of DARPin-F10 efficacy on viral growth curves is based on the total produced virus trajectories, however, both our fitting and validation of kinetic models highlighted that our best kinetic models struggle to fully capture viral production. Overall, it would, perhaps, be useful to not only focus the experiments on total viral production, but also include infectious viral production and cellular trajectories, since it's feasible to imagine that uncoating inhibition may increase the production of DIPs reducing observed infectious fraction.

Despite these difficulties, based on our simple kinetic models of DARPin-F10 effect we can confirm the observation \cite{heldt2013multiscale} that drug interventions targeting viral production are more efficient at achieving reduction in viral titer than interventions targeting primarily targeting viral entry. Additionally, that raises a question of whether uncoating inhibitions counts as rather first or the second, or neither, requiring higher level of precision in kinetic model implementation.
\section{Compartmental model equations}
\label{appendix:compartmentalModelEquations}

All models use the following notation:

\begin{itemize}
\item $T$ - target cells,
\item $E$ - infected cells in eclipse phase,
\item $I$ - productively infected cells,
\item $R$ - removed cells,
\item $V$ - total virus,
\item $V_i$ - infectious virus,
\item $F$ - DARPin-F10,
\item $g$ - rate of target cell generation,
\item $\beta$ - infection rate,
\item $k$ - rate of maturation of eclipse phase cells into productively infected cells,
\item $\delta$ - rate of infected cells death,
\item $c$ - rate of virus clearance,
\item $\tau$ - delay time,
\item $i_{fraction}$ - infectious fraction,
\item $\epsilon_{drug}$ - drug effect,
\item $\phi$ - rate of DARPin-F10 production or depletion, for all the models it is set to 0,
\item $EC_{50}^{Target}$ half maximal effective concentration of target cells,
\item $n^{Target}$ Hill coefficient of target depletion,
\item $IC_{50}^{DARPin-F10}$ half maximal inhibiting concentration of DARPin-F10 cells,
\item $n^{DARPin-F10}$ Hill coefficient of DARPin-F10 inhibition.
\end{itemize}

For all the delay models the history is set to initial conditions, except for total and infectious virus which are multiplied by sigmoid function $S(t)$ to simulate the addition of virus to the cells.

\begin{equation}
S(t) = \frac{1}{1 + exp(-100t)}
\end{equation}

\subsection{TIRVV$_i$}

\begin{equation}
\begin{array}{rcl}
\frac{dT}{dt} &=& - \beta T V_i \\
\frac{dI}{dt} &=& \beta T V_i - \delta I \\
\frac{dR}{dt} &=& \delta I \\
\frac{dV}{dt} &=& p I - c V\\
\frac{dV_i}{dt} &=& i_{fraction} (p I - c V)
\end{array}
\end{equation}

Chronic TIRVV$_i$ includes an additional term $g T$ in target cell equation.

\subsection{TEIRVV$_i$}

\begin{equation}
\begin{array}{rcl}
\frac{dT}{dt} &=& - \beta T V_i \\
\frac{dE}{dt} &=& \beta T V_i - k E \\
\frac{dI}{dt} &=& k E - \delta I \\
\frac{dR}{dt} &=& \delta I \\
\frac{dV}{dt} &=& p I - c V\\
\frac{dV_i}{dt} &=& i_{fraction} (p I - c V)
\end{array}
\end{equation}

Chronic TEIRVV$_i$ includes an additional term $g T$ in target cell equation.

\subsection{TIRVV$_i$ delay}

\begin{equation}
\begin{array}{rcl}
\frac{dT}{dt} &=& - \beta T V_i \\
\frac{dI}{dt} &=& \beta T V_i - \delta I \\
\frac{dR}{dt} &=& \delta I \\
\frac{dV}{dt} &=& p I (t - \tau) - c V\\
\frac{dV_i}{dt} &=& i_{fraction} (p I(t - \tau) - c V)
\end{array}
\end{equation}

Equations are the same for both $\tau_{delay} = f(MOI)$ and $\tau_{delay} = const$, and the value of $\tau$ is supplied as a parameter.

\subsection{T$_{Hill}$IRVV$_i$}

\begin{equation}
\begin{array}{rcl}
\frac{dT}{dt} &=& - \beta \frac{T_0}{1+\exp \big(n^{Target}\log_e(\frac{EC_{50}^{Target}}{T})\big)} V_i \\
\frac{dI}{dt} &=&  \beta \frac{T_0}{1+\exp \big(n^{Target}\log_e(\frac{EC_{50}^{Target}}{T})\big)} V_i - \delta I \\
\frac{dR}{dt} &=& \delta I \\
\frac{dV}{dt} &=& p I - c V\\
\frac{dV_i}{dt} &=& i_{fraction} (p I - c V)
\end{array}
\end{equation}

\subsection{T$_{Hill}$IRVV$_i$ delay}

\begin{equation}
\begin{array}{rcl}
\frac{dT}{dt} &=& - \beta \frac{T_0}{1+\exp \big(n^{Target}\log_e(\frac{EC_{50}^{Target}}{T})\big)} V_i \\
\frac{dI}{dt} &=&  \beta \frac{T_0}{1+\exp \big(n^{Target}\log_e(\frac{EC_{50}^{Target}}{T})\big)} V_i - \delta I \\
\frac{dR}{dt} &=& \delta I \\
\frac{dV}{dt} &=& p I (t - \tau) - c V\\
\frac{dV_i}{dt} &=& i_{fraction} (p I (t - \tau) - c V)
\end{array}
\end{equation}

Equations are the same for both $\tau_{delay} = f(MOI)$ and $\tau_{delay} = const$, and the value of $\tau$ is supplied as a parameter.

\subsection{DARPin-F10 drug effect}

All the models with DARPin-F10 include additional equation:

\begin{equation}
\frac{dF}{dt} = \phi F
\end{equation}

However the DARPin-F10 concentration in our simulations is considered a constant, because we set $\phi = 0$.

Amantadine-like DARPin-F10 models include $(1-\epsilon_{drug})$ multiplier before $\beta$, while neuraminidase inhibitor-like DARPin-F10 include it before parameter $p$. $\epsilon_{drug}$ is calculated as

\begin{equation}
\frac{dF}{dt} = \frac{1}{1 + \exp\big(n^{DARPin-F10} \log_e (\frac{IC_{50}^{DARPin-F10}}{F})\big)}
\end{equation}
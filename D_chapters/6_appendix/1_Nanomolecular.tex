\section{Effective forces on nanomolecular scales}
\label{appendix:nanomolecular}

In classical mechanics an object moving in fluid is affected by several forces:
\begin{itemize}
    \item weight (directed down),
    \item buoyant force (directed up),
    \item drag force (directed opposite the direction of movement).
\end{itemize}

Further, in fluids the Reynolds number is used to determine specific flow patterns and relevance of inertial or viscous forces. In systems with low Reynolds numbers it is sometimes possible to omit inertial forces as fluid dynamics play a much more important role \cite{purcell1977life}.

Additionally, capsid disassembly involves forces exerted by molecular motors and restoring forces of bonds between M1.

Here we estimate sedimentation forces and mechanical forces involved in capsid disassembly, to determine their relevancy to biophysical model.

\subsection{Reynolds number}

Reynolds number is defined as

\begin{equation}
Re = \frac{\text{inertial forces}}{\text{viscous forces}} =\frac{av\rho}{\eta} = \frac{av}{\nu}
\end{equation}

where $a$ is a characteristic size of the moving object, $v$ is its velocity relative to fluid, $\rho$ is fluid density, $\eta$ is fluid viscosity and $\nu$ is kinematic viscosity.

\begin{itemize}
\item For influenza A capsid protein M1 characteristic size is $a = 4 \text{ nm} = 4 \cdot 10^{-7} \text{ cm}$ \cite{shtykova2013structural}.
\item During the simulation individual velocities of the nodes typically are in the range of $10^{-13}$-$10^{-5} \text{ m/s}$. Motor forward velocities are around $10^{-6} \text{ m/s}$ \cite{muller2008tug}. Let us assume for this calculation that $v = 10^{-6} \text{ m/s} = 10^{-4} \text{ cm/s}$.
\item Average density of the cytoplasm (based on horn cells in rats \cite{hartmann1967cytoplasmic}) is about $\rho = 0.3 \text{ pg}/\SI{}{\micro\meter}^3 = 0.3 \text{ g}/\text{cm}^3$ . Alternatively for \textit{E. coli} $\rho = 1.1 \text{ g}/\text{cm}^3$ \cite{loferer1998determination}.
\item Relative viscosity of cytoplasm (cytoplasm versus water) determined by time-resolved microfluorimetry \cite{swaminathan1997photobleaching} is $\frac{\eta_\text{cytoplasm}}{\eta_\text{water}}$ = 1.5 and determined by photobleaching recovery of green fluorescent protein \cite{swaminathan1997photobleaching} is 3.2.
\item Water viscosity is $\eta_\text{water} = 8.9 \cdot 10^{-3} \text{ dyn} \cdot \text{s}/\text{cm}^2 = 8.9 \cdot 10^{-3} \text{ g}/\text{cm} \cdot \text{s} $ at \SI{25}{\degreeCelsius} \cite{IAPWS2008}. So $\eta_\text{cytoplasm} = 3.2 \cdot 8.9 \cdot 10^{-3} \text{ g}/(\text{cm} \cdot \text{s}) = 28.5 \cdot 10^{-3} \text{ g}/(\text{cm} \cdot \text{s})$.
\end{itemize}

Thus, Reynolds number is

\begin{equation}
Re = \frac{av\rho}{\eta} = \frac{4 \cdot 10^{-7} \text{ cm} \cdot 10^{-4} \text{ cm/s} \cdot 0.3 \text{ g}/\text{cm}^3}{28.5 \cdot 10^{-3} \text{ g}/(\text{cm} \cdot \text{s})} = 4.2 \cdot 10^{-10}
\end{equation}

This value is extremely small - for comparison, the Reynolds number for a bacterium moving in the medium is approximately $3\cdot 10^{-5}$ \cite{purcell1977life}, which means that for our system intertia forces do not play any important role, and that what happens to the proteins in question is determined by the forces exerted on them at that specific moment \cite{purcell1977life}.

\subsection{Weight}

The mass of M1 capsid protein is $m_{\text{M1}} = 28 \text{ kDa} = 28 \cdot 1000 \cdot 1.66 \cdot 10^{-27} \text{ kg} = 46.5 \cdot 10^{-24} \text{ kg}$. \cite{shtykova2013structural}

Then the force is

\begin{equation}
F_{\text{weight}} = m_{M1}g = 46.5 \cdot 10^{-24} \text{ kg} \cdot 9.8 \text{ m}/\text{s}^2 = 4.6 \cdot 10^{-22} \text{ N}
\end{equation}

where $g$ is gravitational acceleration.

\subsection{Buoyant force}

\begin{equation}
F_{\text{buoyant}} = \rho_FVg,
\end{equation}

where $\rho_F$ is density of the fluid, $V$ is the volume of the object and $g$ is gravitational acceleration. Using previously established values for characteristic size of M1 and density of cytoplasm we obtain the value for the force

\begin{equation}
\begin{split}
F_{\text{buoyant}} = \rho_{\text{cytoplasm}}V_{M1}g =\\
3000 \text{ kg}/\text{m}^3 \cdot 9.6 \cdot 10^{-26} \text{ m}^3 \cdot 9.8 \text{ m}/\text{s}^2 =\\
2.8 \cdot 10^{-21} \text{ N}
\end{split}
\end{equation}

\subsection{Drag force}

Drag force exerted on spherical objects with very small Reynolds numbers (i.e. very small particles) in a viscous fluid is defined by Stokes' law:

\begin{equation}
F_{\text{drag}} = 6\pi\eta av,
\end{equation}

where $\eta$ is dynamic fluid viscosity, $a$ is the radius of the spherical object, $v$ is the flow velocity relative to the object. Using previously established values

\begin{equation}
\begin{split}
F_{\text{drag}} = 6\pi\eta_{\text{cytoplasm}} av = \\
6 \cdot 3.14 \cdot 2.85 \cdot 10^{-4} \text{ kg}/(\text{m} \cdot \text{s}) \cdot 4 \cdot 10^{-9} \text{ m} \cdot 10^{-6} \text{ m/s} =\\
2.1 \cdot 10^{-17} \text{ N}.
\end{split}
\end{equation}

\subsection{Motor forces}

Stall forces of molecular motors are in orders of pN \cite{muller2008tug}, which leads to resulting tug-of-war forces to be of the same order $F_{\text{motor}} = 10^{-12} \text{ N}$.

\subsection{Spring forces}

In the simplest case spring forces can be calculated using Hooke's law:

\begin{equation}
F_{\text{Hooke}} = k_{\text{spring}}\Delta x.
\end{equation}

At pH of interest the bond spring constant is $k_{\text{spring}} = 2.1 \cdot 10^{-2} \text{ N}/\text{m}$ \cite{li2014ph}

The displacements in the calculation are of the order of the equilibrium length of the spring: $\Delta x = 4 \text{ nm} = 4 \cdot 10^{-9} \text{ m}$

Then the force is

\begin{equation}
\begin{split}
F_{\text{Hooke}} = k_{\text{spring}}\Delta x =\\
2.1 \cdot 10^{-2} \text{ N}/\text{m} \cdot 4 \cdot 10^{-9} \text{ m} = 8.4 \cdot 10^{-11} \text{ N}.
\end{split}
\end{equation}

In our representation of capsid disassembly instead of Hooke's law we are using Morse potential, because it explicitly includes the possibility of breakage:

\begin{equation}
V(\mathbf{r}) = D_{\text{e}} \big(1 - e^{- a(\mathbf{r - r_e})}\big)^2 = D_{\text{e}} \big(1 - e^{- a \Delta \mathbf{r}}\big)^2.
\end{equation}

Here $\mathbf{r}$ is the distance, $\mathbf{r_e}$ is the equilibrium bond distance, $D_{\text{e}}$ is the well depth,

\begin{equation}
a=\sqrt {\frac{k_{\text{e}}}{2D_{\text{e}}}},
\end{equation}

where $k_e$ is the force constant at the minimum of the well. $a$ controls the ``width'' of the potential (the smaller $a$ is, the larger the well).

The restoring force is

\begin{equation}
\mathbf{F}_\text{Morse} = - 2a D_{\text{e}}e^{- a |\Delta \mathbf{x}|}\big(1 - e^{- a |\Delta \mathbf{x}|}\big) \cdot \big(- \frac{\Delta \mathbf{x}}{|\Delta \mathbf{x}|}\big).
\end{equation}

For $D_{\text{e}} = 20 \text{kJ}$ (the energy of hydrogen bond \cite{mcnaught1997compendium}) and $k_{\text{e}} = k_{\text{spring}} = 2.1 \cdot 10^{-2} \text{ N}/\text{m}$

\begin{equation}
\begin{split}
a = \sqrt {\frac{2.1 \cdot 10^{-2} \text{ N}/\text{m}}{2 \cdot 2 \cdot 10^{4} \text{J}}}\\
= \sqrt {\frac{2.1 \cdot 10^{-2} (\text{ kg} \cdot \text{m} \cdot \text{s}^-2)/\text{m}}{2 \cdot 2 \cdot 10^{4} (\text{ kg} \cdot \text{m}^2 \cdot \text{s}^-2)}} = 7.2 \text{ m}^{-1}
\end{split}
\end{equation}

\begin{equation}
e^{- a |\Delta \mathbf{x}|} = e^{- 7.2 \text{ m}^{-1} \cdot 4 \cdot 10^{-9} \text{ m}} = 1
\end{equation}

\begin{equation}
1 - e^{- a |\Delta \mathbf{x}|} = 1 - e^{- 7.2 \text{ m}^{-1} \cdot 4 \cdot 10^{-9} \text{ m}} = 2.9 \cdot 10^{-9}
\end{equation}

\begin{equation}
\begin{split}
F_\text{Morse} = 2aD_{\text{e}}e^{- a |\Delta \mathbf{x}|}\big(1 - e^{- a |\Delta \mathbf{x}|}\big) =\\
2 \cdot 7.2 \text{ m}^{-1} \cdot 2 \cdot 10^{4} \text{J} \cdot 1 \cdot 2.9 \cdot 10^{-9} = 8.4 \cdot 10^{-4} \text{ N}.
\end{split}
\end{equation}

This estimate is a little high, since during our first calculation steps active spring forces are approximately $10^{-21} \text{ N}$, but reaches up to $10^{-9} \text{ N}$ during the course of the calculation.

Based on the estimates of the force values in our model we focus on spring and motor forces, because sedimentation forces acting on the proteins are negligibly small.


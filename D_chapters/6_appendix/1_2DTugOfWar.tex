\section{Derivation of two dimensional Tug-of-War equations}
\label{appendix:2DTugOfWar}

First, let us examine the one-dimensional case, where all the motors walk along the microtubule. Analogously to the two-motor scenario \cite{gennerich2007force, muller2008tug, norstrom2010unconventional}, we write a force balance.

\begin{equation}
n_kf_k + n_{m+}f_{m+} = -n_df_d - n_{m-}f_{m-} = f_C(n_k, n_d, n_{m+}, n_{m-})
\label{eq:forceBalance}
\end{equation}

where $n_k$, $n_d$, $n_{m+}$, $n_{m-}$ are motor numbers and $f_k$, $f_d$, $f_{m+}$, $f_{m-}$ are forces of kinesin, dynein, plus- and minus-end myosin motors, respectively. The cargo force is determined by the condition that all motors move with the same velocity $v_C$ , as given by

\begin{equation}
v_C(n_k, n_d, n_{m+}, n_{m-}) = v_k(f_k) = v_{m+}(f_{m+}) = - v_d(f_d) = - v_{m-}(f_{m-})
\end{equation}

With this definition follows for the left part of the force balance Equation \ref{eq:forceBalance}:

\begin{equation}
f_C(n_k, n_d, n_{m+}, n_{m-}) = n_kf_k + n_{m+}f_{m+} = n_kf_k\big(1 + \frac{n_{m+}f_{m+}}{n_kf_{k}}\big)
\end{equation}

This gives us an expression for the kinesin force:

\begin{equation}
f_k = \frac{f_C(n_k, n_d, n_{m+}, n_{m-})}{n_k\big(1 + \frac{n_{m+}f_{m+}}{n_kf_{k}}\big)}
\end{equation}

Analogously, for the right part and the dynein force, we obtain:

\begin{equation}
f_d = -\frac{f_C(n_k, n_d, n_{m+}, n_{m-})}{n_d\big(1 + \frac{n_{m-}f_{m-}}{n_df_{d}}\big)}
\end{equation}

For simplicity, we assume here that the ratio of motor forces moving in the same direction is proportionate to the ratio of their stall forces (indicated by subscript S):

\begin{equation}
\frac{f_{m+}}{f_k} \propto \frac{f_{Sm+}}{f_{Sk}}, 
\frac{f_{m-}}{f_d} \propto \frac{f_{Sm-}}{f_{Sd}}
\end{equation}

This allows us to write the expressions for kinesin and dynein forces as follows

\begin{equation}
f_k = \frac{f_C(n_k, n_d, n_{m+}, n_{m-})}{n_kC_k(n_{m+}, n_k)}
\end{equation}

\begin{equation}
f_d = -\frac{f_C(n_k, n_d, n_{m+}, n_{m-})}{n_dC_d(n_{m-},n_d)}
\end{equation}

where

\begin{equation}
C_k(n_{m+}, n_k) = 1 + \frac{n_{m+}f_{Sm+}}{n_kf_{Sk}}
\end{equation}

\begin{equation}
C_d(n_{m-},n_d) = 1 + \frac{n_{m-}f_{Sm-}}{n_df_{Sd}}
\end{equation}

We know that the velocity of a motor can be expressed as a function of the load force \cite{muller2008tug}:

\begin{equation}
v(f) = \left\{ \begin{aligned}
v_{forward} (1 - \frac{|f|}{f_S}), for 0 \leq  f \leq f_S\\
v_{backward} (1 - \frac{f}{f_S}), for f > f_S
\end{aligned}\right.
\end{equation}

Now we can use the expression for the cargo velocity to derive:

\begin{equation}
v_C(n_k, n_d, n_{m+}, n_{m-}) = v_k(f_k) = - v_d(-f_d)
\end{equation}

\begin{equation}
\begin{split}
v_C(n_k, n_d, n_{m+}, n_{m-}) = v_{0k}\big(1-\frac{f_k}{f_{Sk}}\big) =\\
v_{0k}\big(1-\frac{1}{f_{Sk}}\frac{f_C(n_k, n_d, n_{m+}, n_{m-})}{n_kC_k(n_{m+}, n_k)} = \\
v_{0k}-\frac{v_{0k}}{f_{Sk}}\frac{f_C(n_k, n_d, n_{m+}, n_{m-})}{n_kC_k(n_{m+}, n_k)}
\end{split}
\end{equation}

\begin{equation}
\begin{split}
v_C(n_k, n_d, n_{m+}, n_{m-}) = -v_{0d}\big(1-\frac{f_d}{f_{Sd}}\big) =\\ -v_{0d}\big(1-\frac{1}{f_{Sd}}\frac{f_C(n_k, n_d, n_{m+}, n_{m-})}{n_dC_d(n_{m-},n_d)}\big) = \\
-\big(v_{0d}-\frac{v_{0d}}{f_{Sd}}\frac{f_C(n_k, n_d, n_{m+}, n_{m-})}{n_dC_d(n_{m-},n_d)}\big)
\end{split}
\end{equation}

\begin{equation}
\begin{split}
v_C(n_k, n_d, n_{m+}, n_{m-}) =\\ v_{0k}-\frac{v_{0k}}{f_{Sk}}\frac{f_C(n_k, n_d, n_{m+}, n_{m-})}{n_kC_k(n_{m+}, n_k)} =
-\big(v_{0d}-\frac{v_{0d}}{f_{Sd}}\frac{f_C(n_k, n_d, n_{m+}, n_{m-})}{n_dC_d(n_{m-},n_d)}\big)
\end{split}
\end{equation}

By shuffling the second and third terms in this equation, we get the expression for the cargo force

\begin{equation}
\begin{split}
v_{0k} + v_{0d}  = \\
\frac{v_{0k}}{f_{Sk}}\frac{f_C(n_k, n_d, n_{m+}, n_{m-})}{n_kC_k(n_{m+}, n_k)} + \frac{v_{0d}}{f_{Sd}}\frac{f_C(n_k, n_d, n_{m+}, n_{m-})}{n_dC_d(n_{m-},n_d)}
\end{split}
\end{equation}

\begin{equation}
\begin{split}
v_{0k} + v_{0d}  = \\
f_C(n_k, n_d, n_{m+}, n_{m-}) \big( \frac{v_{0k}}{f_{Sk}n_kC_k(n_{m+}, n_k)} + \frac{v_{0d}}{f_{Sd}n_dC_d(n_{m-},n_d)} \big)
\end{split}
\end{equation}

\begin{equation}
\begin{split}
v_{0k} + v_{0d}  = \\
f_C(n_k, n_d, n_{m+}, n_{m-}) \big( \frac{v_{0k}f_{Sd}n_dC_d(n_{m-},n_d) + v_{0d}f_{Sk}n_kC_k(n_{m+}, n_k)}{f_{Sk}n_kC_k(n_{m+}, n_k)f_{Sd}n_dC_d(n_{m-},n_d)} \big)
\end{split}
\end{equation}

\begin{equation}
\begin{split}
f_C(n_k, n_d, n_{m+}, n_{m-}) = \\
(v_{0k} + v_{0d})\frac{f_{Sk}n_kC_k(n_{m+}, n_k)f_{Sd}n_dC_d(n_{m-},n_d)}{v_{0k}f_{Sd}n_dC_d(n_{m-},n_d) + v_{0d}f_{Sk}n_kC_k(n_{m+}, n_k)}
\end{split}
\end{equation}

Now, for convenience we rewrite the equation:
\begin{equation}
\begin{split}
f_C(n_k, n_d, n_{m+}, n_{m-}) = \\
(v_{0k} + v_{0d})\frac{f_{Sk}n_kC_k(n_{m+}, n_k)f_{Sd}n_dC_d(n_{m-},n_d)}
{v_{0k}f_{Sd}n_dC_d(n_{m-},n_d) (1 + \frac{v_{0d}f_{Sk}n_kC_k(n_{m+}, n_k)}{v_{0k}f_{Sd}n_dC_d(n_{m-},n_d)})}
\end{split}
\end{equation}

\begin{equation}
\begin{split}
f_C(n_k, n_d, n_{m+}, n_{m-}) = \\
\frac{v_{0k}f_{Sk}n_kC_k(n_{m+}, n_k)f_{Sd}n_dC_d(n_{m-},n_d) + v_{0d}f_{Sk}n_kC_k(n_{m+}, n_k)f_{Sd}n_dC_d(n_{m-},n_d)}
{v_{0k}f_{Sd}n_dC_d(n_{m-},n_d) (1 + \frac{v_{0d}f_{Sk}n_kC_k(n_{m+}, n_k)}{v_{0k}f_{Sd}n_dC_d(n_{m-},n_d)})}
\end{split}
\end{equation}

\begin{equation}
\begin{split}
f_C(n_k, n_d, n_{m+}, n_{m-}) = \\
\frac{f_{Sk}n_kC_k(n_{m+}, n_k) + \frac{v_{0d}f_{Sk}n_kC_k(n_{m+}, n_k)f_{Sd}n_dC_d(n_{m-},n_d)}{v_{0k}f_{Sd}n_dC_d(n_{m-},n_d)}}
{(1 + \frac{v_{0d}f_{Sk}n_kC_k(n_{m+}, n_k)}{v_{0k}f_{Sd}n_dC_d(n_{m-},n_d)})}
\end{split}
\end{equation}

\begin{equation}
\begin{split}
f_C(n_k, n_d, n_{m+}, n_{m-}) = \\
f_{Sk}n_kC_k(n_{m+}, n_k) \frac{1}{(1 + \frac{v_{0d}f_{Sk}n_kC_k(n_{m+}, n_k)}{v_{0k}f_{Sd}n_dC_d(n_{m-},n_d)})} + \\
f_{Sd}n_dC_d(n_{m-},n_d)\frac{\frac{v_{0d}f_{Sk}n_kC_k(n_{m+}, n_k)}{v_{0k}f_{Sd}n_dC_d(n_{m-},n_d)}}
{(1 + \frac{v_{0d}f_{Sk}n_kC_k(n_{m+}, n_k)}{v_{0k}f_{Sd}n_dC_d(n_{m-},n_d)})}
\end{split}
\end{equation}

\begin{equation}
\begin{split}
f_C(n_k, n_d, n_{m+}, n_{m-}) = \\
f_{Sk}n_kC_k(n_{m+}, n_k) \frac{1}{(1 + \frac{v_{0d}f_{Sk}n_kC_k(n_{m+}, n_k)}{v_{0k}f_{Sd}n_dC_d(n_{m-},n_d)})} + \\
f_{Sd}n_dC_d(n_{m-},n_d)\frac{1+\frac{v_{0d}f_{Sk}n_kC_k(n_{m+}, n_k)}{v_{0k}f_{Sd}n_dC_d(n_{m-},n_d)}-1}
{(1 + \frac{v_{0d}f_{Sk}n_kC_k(n_{m+}, n_k)}{v_{0k}f_{Sd}n_dC_d(n_{m-},n_d)})}
\end{split}
\end{equation}

to get the following expression for cargo force as a function of motor abundances

\begin{equation}
\begin{split}
f_C(n_k, n_d, n_{m+}, n_{m-}) = \\
f_{Sk}n_kC_k(n_{m+}, n_k) \lambda(n_k, n_d, n_{m+}, n_{m-}) + \\
f_{Sd}n_dC_d(n_{m-},n_d)( 1 - \lambda(n_k, n_d, n_{m+}, n_{m-}) )
\end{split}
\end{equation}

where

\begin{equation}
\lambda(n_k, n_d, n_{m+}, n_{m-}) = \frac{1}{1 + \frac{v_{0d}f_{Sk}n_kC_k(n_{m+}, n_k)}{v_{0k}f_{Sd}n_dC_d(n_{m-},n_d)}}
\end{equation}

After computing the cargo force, we can calculate the expressions for the kinesin and dynein forces as described in (2.27-2.30). Using those expressions we get the expressions for myosin forces from stall forces using (2.26)

\begin{equation}
f_{m+} = \frac{f_{Sm+}f_k}{f_{Sk}}, 
f_{m-} =  \frac{f_{Sm-}f_d}{f_{Sd}}
\end{equation}

In the two-dimensional case, dyneins and kinesins would still be walking along the microtubule, such that only the expressions for the myosins would change. Myosin forces are involved in the calculation of $C_k(n_{m+}, n_k)$ and $C_d(n_{m-},n_d)$, but if we examine the formulas, we see that $n_{m\pm}f_{Sm\pm}$ are simply cumulative forces created by plus- and minus-end myosin motors. Thus, we can easily substitute these terms for $\sum_{i}\hat{f}^i_{Sm\pm}$, where $\hat{f}^i_{Sm\pm}$ is a projection of the stall force of a myosin motor $i$ on the direction of the microtubule. Similarly, expressions for myosin forces are computed individually for each motor with $f_{Sm\pm}$ being substituted by the projection $\hat{f}^i_{Sm\pm}$.

For simplicity, here we disregard tug-of-war between myosins in the direction perpendicular to the microtubule. However, this interaction can be computed and included into simulations similarly, by using the results of \cite{gennerich2007force, muller2008tug, norstrom2010unconventional} for two motors:

\begin{equation}
\tilde{f}_C(n_{m+}, n_{m-}) = n_{m+}f_{Sm+}\tilde{\lambda}(n_{m+}, n_{m-}) + n_{m-}f_{Sm+}(1 - \tilde{\lambda}(n_{m+}, n_{m-}))
\end{equation}

where

\begin{equation}
\tilde{\lambda}(n_{m+}, n_{m-}) = \frac{1}{1+\frac{n_{m+}f_{Sm+}v_{0m-}}{n_{m-}f_{Sm-}v_{0m+}}}
\end{equation}
 
Like before, we substitute $n_{m\pm}f_{Sm\pm}$ for $\sum_{i}\tilde{f}^i_{Sm\pm}$, where $\tilde{f}^i_{Sm\pm}$ are projections of the stall force of motor i in the direction perpendicular to the microtubule. Knowing $\tilde{f}_C(n_{m+}, n_{m-})$, we can compute the motor forces as:

\begin{equation}
\tilde{f}_{m+} = \frac{\tilde{f}_C(n_{m+}, n_{m-})}{n_{m+}}
\end{equation}

\begin{equation}
\tilde{f}_{m-} = \frac{\tilde{f}_C(n_{m+}, n_{m-})}{n_{m-}}
\end{equation}
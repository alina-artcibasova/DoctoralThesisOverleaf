\chapter{Concluding Remarks and Outlook}
\label{ch:Conclusion}

\dictum[Elena Pokatovich,\\Economics lectures at MIPT]{%

\begin{otherlanguage}{russian}
Мы не будем это обсуждать, подумаете об этом на досуге... если, конечно, у вас есть микроэкономический досуг.
\end{otherlanguage}

We are not going to discuss this, think about it during your free time... if, of course, you have a free time dedicated to microeconomics.
}%
\vskip 1em

In this thesis, we presented three computational models describing HDAC6-mediated influenza A virus uncoating with different levels of detail. We use them to analyze underlying host-viral interactions, and showcase how such models can be combined to characterize influenza infection response to drug-like compounds in cellular culture.

In Chapter \ref{ch:TugOfWar} we used a biophysical model to describe simplified HDAC6-mediated capsid breakage, and to demonstrate the feasibility of two-dimensional tug-of-war between actin filament and microtubule processive molecular motors as a method for capsid breakage force generation. Our results highlight the importance of specific geometries in force generation processes. However, we rely on the simplified geometry to describe the fusion pore and capsid protein M1-M1 bonds. Further investigation into M1 protein structure and oligomerisation patterns \cite{zhang2012dissection} may yield additional insights into the uncoating process.

We only briefly examine the influence of other microtubular motors, such as kinesin on influenza uncoating. Recent inhibition studies \cite{cho2020selective, kim2021kif11} suggest that they may assist capsid breakage through cargo-mediated force balance. Further, given the evidence for their involvement in other viral infections \cite{strunze2011kinesin, greber2019adenovirus, lukic2014hiv}, we can't disregard the possibility that other types of structural instabilities (beyond fusion pores) may lead to capsid breakage during uncoating.

In Chapter \ref{ch:ReactionModels} we used a combined biochemical-biophysical models to describe a simplified HDAC6 complex formation, required for molecular motor recruitment during influenza uncoating. We used "Viral Ub" model variant with competitive Ub assistance in HDAC6-M1 binding to demonstrate that such mechanism is unable to recruit a sufficient number of motors to achieve capsid breakage. We used "Asymmetric" and "Symmetric" model variants to identify the essential role of unanchored ubiquitin chains in bridging HDAC6 and the actomyosin network to mediate uncoating, and showed that they can robustly predict uncoating efficiency \textit{in vivo}. Despite the uncertainty in model parameters, and the minimalist description of the core underlying mechanisms, our models robustly predict influenza virus uncoating in a variety of perturbed conditions. Further, the discrepancies in model predictions highlight the processes which may require further investigation, such as for example, pleiotropic effects of nocadazole on cellular processes, or HDAC6-independent pathways by which influenza may achieve uncoating. 

An intriguing open question is if, and to what extent the tug-of-war uncoating mechanism transfers to other viruses. Adenovirus uncoating may have the closest similarity, with the nuclear pore complex (NPC) acting as a static ‘hold’ that facilitates disruptive force generation by microtubule-associated kinesins (where the NPC-equivalent are dyneins for IAV) \cite{flatt2019adenovirus, greber2019adenovirus}. For HIV-1 the picture is more complex: initial opening of the capsid is the rate-limiting step for uncoating \cite{marquez2018kinetics}, but it is unclear if \textit{in vivo} the initial breach is caused by a tug-of-war mechanism \cite{rawle2018toward}, pressure increase in the capsid due to reverse transcription \cite{rankovic2017reverse}, or a dedicated uncoating receptor akin to the case for enteroviruses \cite{zhao2019human}. Intriguingly, the proposed uncoating receptor for HIV-1 is the $\beta$-karyopherin transportin-1 (TRN-1/TNPO1) \cite{fernandez2019transportin}, the same host protein that induces the debundling of IAV ribonucleoproteins after capsid breakage \cite{miyake2019influenza, yamauchi2020influenza}. Differences between HIV-1 and IAV uncoating may simply result from, for example, transcription initiation of viral genes in the cytoplasm and nucleus, respectively. We argue that such parallels nevertheless warrant more detailed experimental \textit{in vivo} probing of potential tug-of-war mechanisms in HIV-1 uncoating, preferably integrated with quantitative mechanistic modeling as shown here.

In Chapter \ref{ch:DARPin} we expanded our biochemical-biophysical model to include the drug-like compound DARPin-F10, which inhibits Ub binding to HDAC6-ZnF. Using this model's simulation results we were able to predict the influenza uncoating dose response to DARPin-F10, and coupled with available experimental data - corresponding effective concentrations of DARPin-F10. The use of mechanistic biochemical-biophysical models as predictors for dose response is conceptually novel. This approach relies on existing knowledge of protein interactions and could allow to resolve an ambiguity surrounding novel antiviral compounds and their mechanism of action. Further, we used the available literature data \cite{rudiger2019multiscale, schulze2009infection} to assess existing non-structured kinetic models of influenza infection. We attempted to expand these models by analyzing functional relationship between influenza infection conditions and modelling parameters. We observe that the log-linear manner of degradation of infectious fraction over time may prove a useful descriptor, and hope for further experimental investigation. Using our best non-structured kinetic model variants we modeled two kinds of drug-like effects for DARPin-F10. Despite our expectations, the neuraminidase inhibitor-like model, which modifies viral production, compared better with experimental data than amantadine-like model, which modifies the rate of infection. Such discrepancy may indicate that HDAC6-ZnF inhibition by DARPin-F10 leads to further downstream inhibition in virus production.

Such biophysical-biochemical models can be applied to other processes, beyond viral uncoating. For example, polymeric nanocompartments can be induced to self-organize through use of DNA-strands \cite{liu2016dna}. Such systems could be used to create artificial organelles serving as molecular factories. Using DNA hybridization models \cite{karamasioti2019computational}, one could predict the binding and resulting elasticity of DNA bridges, and then combine them with soft body models of polymeric nanocompartments to simulate structural behaviors of such complexes. In general, pattern-specific models of biological membranes \cite{cheng2019biological} with membrane proteins \cite{ayton2009systematic}, or models of cellular filaments and molecular motors \cite{chen2019remote} can describe system state transitions, such as capsid breakage we use here, allowing to make further mechanistic predictions about biological outcomes. 

Overall, this study demonstrates how layered mechanism-based mathematical models can bridge the gaps between different fields of knowledge about complex multiscale systems, and inform future enquiry.
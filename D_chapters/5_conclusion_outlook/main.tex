\chapter{Concluding Remarks and Outlook}
\label{ch:Conclusion}

\dictum[Elena Pokatovich,\\Economics lectures at MIPT]{%

\begin{otherlanguage}{russian}
Мы не будем это обсуждать, подумаете об этом на досуге... если, конечно, у вас есть микроэкономический досуг.
\end{otherlanguage}

We are not going to discuss this, think about it during your free time... if, of course, you have a free time dedicated to microeconomics.
}%
\vskip 1em

In this thesis, we presented three computational models describing HDAC6-mediated influenza A virus uncoating with different levels of detail. We focused on their application to the analysis of underlying host-viral interactions, and how they can be combined to characterize influenza infection response to drug-like compound in cellular culture.

In Chapter \ref{ch:TugOfWar} we used a biophysical model to describe simplified HDAC6-mediated capsid breakage, and to demonstrate the feasibility of two-dimensional tug-of-war between molecular motors as a method for capsid breakage force generation. Our results highlight the importance of specific geometries in force generation processes. We do however rely on simplified geometry to describe the fusion pore and capsid protein M1-M1 bonds. Further investigation into M1 protein structure and oligomerisation patterns \cite{zhang2012dissection} may yield additional insights into uncoating process. We only briefly examine the influence of other microtubular motors, such as kinesin \cite{cho2020selective, kim2021kif11}. We do, however, believe that they may assist capsid breakage through cargo-mediated force balance, and therefore encourage further investigation.

In Chapter \ref{ch:ReactionModels} we used a combined biochemical-biophysical model to describe a simplified HDAC6 complex formation, required for molecular motor recruitment during influenza uncoating. We used this model to identify the essential role of unanchored ubiquitin chains in bridging HDAC6 and the actomyosin network to mediate uncoating, and showed that it can robustly predict uncoating efficiency \textit{in vivo}. Despite the uncertainty in model parameters, and minimalist description of the core underlying mechanisms, our model robustly predicts influenza virus uncoating in a variety of perturbed conditions. Further, the discrepancies in model predictions highlight the processes which may require further investigation, such as, for example, pleiotropic effects of nocadazole on cellular processes, or HDAC6-independent pathways by which influenza may achieve uncoating. 

In Chapter \ref{ch:DARPin} we expanded our biochemical-biophysical model to include drug-like compound DARPin-F10, which inhibits Ub binding to HDAC6-ZnF. Using this model's simulation results we were able to predict influenza uncoating dose response to DARPin-F10, and coupled with available experimental data - corresponding effective concentrations of DARPin-F10. Use of mechanistic biochemical-biophysical models as predictors for dose response is conceptually novel. This approach relies on existing knowledge of protein interactions and could allow to resolve an ambiguity surrounding novel antiviral compounds and their mechanism of action. Further, we used the available literature data \cite{rudiger2019multiscale, schulze2009infection} to assess existing non-structured kinetic models of influenza infection. We attempted to expand these models by analyzing functional relationship between influenza infection conditions and modelling parameters. We observe that the log-linear manner of degradation of infectious fraction over time may prove a useful descriptor, and hope for further experimental investigation. 

Overall, this study demonstrates how layered mechanism-based mathematical models can bridge the gaps between different fields of knowledge about complex multiscale systems, and inform future enquiry.
\section{Introduction}

To infect a new host cell, eukaryotic viruses have to successfully complete a sequence of steps that starts with cell entry, often via endocytic pathways, and eventually delivers a functional viral genome at an appropriate intracellular location (depending on the type of virus). One critical step in this sequence is uncoating, the disassembly of the protective outer protein shell (capsid), which enables release of the DNA or RNA and associated proteins from the viral core. Uncoating has long been viewed as a passive process that is pre-programmed during viral assembly: while transiting through the host cell’s different milieus during transport away from the membrane, a metastable viral core experiences triggers, such as acidification in late endosomes, that successively weaken the capsid \cite{marsh2006virus}. Capsid breakage and genome release may also be aided by high internal pressure exerted by the confined, tightly packaged DNA or RNA \cite{brandariz2019pressure}, or by host receptors that destabilize the capsid \cite{zhao2019human}. Detailed mechanistic studies of uncoating are sparse because uncoating is a dynamic, transient phenomenon that is hard to measure experimentally \textit{in vivo} and to replicate \textit{in vitro}. For example, while increased internal pressure due to reverse transcription can uncoat human immunodeficiency virus 1 (HIV-1) \textit{in vitro} \cite{rankovic2017reverse}, \textit{in vivo} interactions of viral capsid and host proteins appear critical for the rate-limiting, initial breakage \cite{marquez2018kinetics, rawle2018toward}. \textit{In vivo} studies, including genetic and biochemical evidence, reveal uncoating – instead of being pre-programmed – as an active process relying on complex, dynamic interactions between viral and host factors, in which the host cell’s cytoskeleton and its associated motor proteins play prominent roles \cite{greber2019adenovirus, helenius2018virus, james2018human, walsh2019exploitation}.

However, the specific roles of the cytoskeleton and of molecular motors remain largely unclear. The prominent hypothesis states that these host components mediate the correct spatio-temporal positioning of the capsid for uncoating. For example, single-particle studies in live cells show that uncoating is spatially and temporally tightly controlled to release the viral genome in the perinuclear region for influenza A virus (IAV) \cite{qin2019real} and in the cytoplasm for HIV-1 \cite{francis2016time}. In this context, the attachment of motor proteins with opposing directionalities to the capsid could be important for efficient virus transport, akin to the transport of intracellular organelles \cite{kural2005kinesin}.

Motors attached to the viral protein capsid could also exert a more direct influence on uncoating by generating forces that mechanically break the capsid in a tug-of-war mechanism. Direct experimental evidence for a tug-of-war comes from adenovirus (AdV), which binds to the nuclear pore complex (NPC) as static ‘hold’ and indirectly to kinesin-1 motors, leading to disruption of both the viral capsid and the NPC \cite{greber2019adenovirus, strunze2011kinesin}. The disassembly patterns also resemble those of mechanically stressed AdV particles \textit{in vitro} leading to genome release \cite{ortega2015fluorescence, ortega2013monitoring}. In addition, it has been hypothesized that different types of motors attached to the same virus particle could uncoat the capsid in a tug-of-war mechanism. For HIV-1, capsid CA proteins can bind to dynein and kinesin-1 molecular motors via cellular adaptor proteins \cite{carnes2018hiv, lukic2014hiv, malikov2017localized}, but it is unknown if they attach to the same particle and if the motors exert sufficient forces \cite{malikov2017localized}. Similarly, IAV exploits the host cell’s aggresome pathway for uncoating, mediated by the recruitment of histone deacetylase 6 (HDAC6) and of molecular motors such as dynein and myosin 10 to the viral capsid \cite{banerjee2014influenza}. These findings are consistent with a tug-of-war mechanism of uncoating that involves different motor types walking on different filaments, but we do not know if such a mechanism exists and what the precise processes involved would be.

To determine the feasibility of such a mechanism, we created a biophysical model which incorporates the viral capsid composed of M1 proteins, the host cytoskeleton, and molecular motors. It shows that a tug-of-war mechanism is physically possible with interacting motors, given their known characteristics.
\section{Discussion}

Our biophysical model describes a predefined configuration of molecular motors which does not change over over time during the simulation. Thus, reported capsid break times (Figure \ref{figure:BreakDuration}) are a simplification, as system will require time to dynamically assemble into those predefined configurations. If those predefined molecular motor configurations are achieved through sequential attachment of fewer molecular motors to the capsid, it's possible that in a real biological system capsid breakage happens for non-optimal motor configurations with fewer attached motors, simply by the virtue of shorter assembly time. Perhaps, a use of stochastic model including motor attachment and detachment dynamics could provide accurate break time estimates, but our ODE model has an advantage of simplicity, and mostly relies on protein characteristics avaiailable in the literature.

For simplicity, in our model we assume that in general case each M1 capsid protein maintains 4 bonds of the same strength with others. \textit{In vitro} studies of M1 indicate that N- and C-terminal M1 capsid protein domains display different pH-dependant oligomerisation behaviors \cite{zhang2012dissection}. Perhaps, a detailed protein structure models of influenza virus capsid may provide a more accurate estimates for bond stiffness and dissociation energy.

Tug-of-war mechanism \cite{hancock2014bidirectional} has been used in mathematical models of bidirectional cargo transport (\cite{muller2008tug}), and in some cases, specifically, nonenveloped virus transport \cite{gazzola2009stochastic} along the microtubules. Both endosomal and nonenveloped virus transport relies on structural integrity of the cargo. However, for influenza virus structural integrity falls under scrutiny. During the endosomal maturation pH-controlled conformational changes in influenza capsid result in decrease of capsid layer stiffness, while outer lipid bilayer would still maintain it's integrity \cite{li2014ph}. Fusion pore formation event would thus expose the the vulnerable capsid layer. In this scenario tug-of-war assumptions of full force balance would only hold for as long as exposed capsid layer stays intact, allowing us to use tug-of-war as a method of force generation in our model.

Our results with respect to dynein capsid breakage without myosin 10 assistance (Figure \ref{figure:fluMassSpring}) indicate an important role specific geometries play in force generation. In our model we do not examine the influence of kinesin motors onto molecular tug of war. Recent evidence that kinesin inhibitors suppress influenza virus replication \cite{cho2020selective, kim2021kif11}, indicates that transport focused tug-of-war between endosomally attached molecular motors kinesin and dynein contributes additionally to eventual capsid breakage, however, it seems unlikely to break the capsid  without assistance of capsid attached myosin 10 and dynein motors. Several capsid attached dynein motors may also generate opposing forces for kinesin motors, but the size of the fusion pore compared to the overall size of the virus particles, and further characteristic size of late endosomes make it logistically difficult. However, such a possibility provides an intriguing possibility for future research, and can be falsified by an uncoating experiment with kinesin inhibitors.

Another simplifying assumption we introduce into our model is that the fusion pore is positioned perfectly to allow dynein motor access to the exposed capsid proteins. We are not aware of any way for the influenza virus to "sense" molecular motor facilitated movement, and thus have any preference with regards to initiating fusion pore. This, ultimately could lead to situations where actual tug-of-war between molecular motors is rather three-dimensional, further indicating preferential reliance on actomyosin network for capsid breakage force generation.
\section{Discussion}

Our biophysical model describes predefined configurations of molecular motors which do not change over over time during the simulation. Thus, observed capsid break times (Figure \ref{figure:PersistentBreaks}, \ref{figure:BreakDuration}) are a simplification, as the system will require time to dynamically assemble into those predefined configurations. Under the assumption that those predefined molecular motor configurations are feasible within a cell, they would be achieved through sequential attachment of molecular motors to the capsid, and thus would progress through other configurations with fewer number of motors. Therefore, it is likely that in a real biological system capsid breakage happens for non-optimal motor configurations with fewer attached motors, simply by the virtue of shorter assembly time. Use of stochastic modelling including motor attachment and detachment dynamics could provide accurate break time estimates, but for our purposes we chose an ODE model as it has an advantage of simplicity and, for a moment, allows us to ignore the feasibility of these predefined motor configurations (Chapter \ref{ch:ReactionModels}).

Our results with respect to dynein capsid breakage without myosin 10 assistance (Figure \ref{figure:fluMassSpring}) indicate an important role specific geometries play in force generation. However, in our model we assume that each M1 capsid protein maintains 4 bonds of the same strength (except for edge nodes) with other capsid proteins. \textit{In vitro} studies of M1 indicate that N- and C-terminal M1 capsid protein domains display different pH-dependant oligomerisation behaviors \cite{zhang2012dissection}. Perhaps, detailed protein structure models of influenza virus capsid may provide an insight into realistic influenza capsid geometry and more accurate estimates for bond stiffness and dissociation energy.

Tug-of-war mechanisms \cite{hancock2014bidirectional} have been used in mathematical models of bidirectional cargo transport \cite{muller2008tug} and, specifically, nonenveloped virus transport \cite{gazzola2009stochastic} along the microtubules. Both endosomal and nonenveloped virus transport rely on structural integrity of the cargo. However, for an influenza virus structural integrity falls under scrutiny. During the endosomal maturation pH-controlled conformational changes in the influenza capsid result in a decrease of capsid layer stiffness, while the outer lipid bilayer would still maintain its integrity \cite{li2014ph}. A fusion pore formation event would thus expose the vulnerable capsid layer. In this scenario, tug-of-war assumptions of force balance would only hold for as long as the exposed capsid layer stays intact (Figure \ref{figure:BreakDuration}), allowing us to use a tug-of-war mechanism rather as a method of force generation.

In our model we do not examine the influence of kinesin motors onto molecular tug-of-war. The recent evidence that kinesin inhibitors suppress influenza virus replication \cite{cho2020selective, kim2021kif11}, indicates that transport focused tug-of-war between endosome-attached molecular motors kinesin and dynein may contribute additionally to eventual capsid breakage. However, it seems unlikely to break the capsid  without assistance of capsid attached myosin 10 and dynein motors. Several capsid-attached dynein motors may also generate opposing forces for kinesin motors, but the size of the fusion pore \cite{lee2010architecture} compared to the size of the virus particles, and further the characteristic size of endosomes \cite{ganley2004rab9} make this logistically difficult.

For simplicity, we assume that the fusion pore is positioned perfectly to allow dynein motor access to the exposed capsid proteins. To allow such preference with regards to location of fusion pore site, an influenza virus would have to be able to "sense" molecular motor facilitated movement. We are not aware of any mechanism which could support it. With that in mind, the actual mechanism of tug-of-war may rather be three-dimensional. But as indicated above, the virus relies on the force generation at the fusion pore, which further suggests preferential reliance on actomyosin network for capsid breakage force generation.
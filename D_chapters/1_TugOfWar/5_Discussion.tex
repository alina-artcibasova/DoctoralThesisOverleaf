\section{Discussion}

Tug-of-war mechanism \cite{hancock2014bidirectional} has been used in mathematical models of bidirectional cargo transport (\cite{muller2008tug}), and in some cases, specifically, nonenveloped virus transport \cite{gazzola2009stochastic} along the microtubules. Both endosomal and nonenveloped virus transport relies on structural integrity of the cargo. However, for influenza virus structural integrity falls under scrutiny. During the endosomal maturation pH-controlled conformational changes in influenza capsid result in decrease of capsid layer stiffness, while outer lipid bilayer would still maintain it's integrity \cite{li2014ph}. Fusion pore formation event would thus expose the the vulnerable capsid layer. In this scenario tug-of-war assumptions of full force balance would only hold for as long as exposed capsid layer stays intact, allowing us to use tug-of-war as a method of force generation in our model.

It is possible that transport focused tug-of-war between endosomally attached molecular motors kinesin and dynein contributes additionally to eventual capsid breakage, however, it seems unlikely to break the capsid  without assistance of capsid attached myosin 10 and dynein motors. Several capsid attached dynein motors may also generate opposing forces for kinesin motors, but the size of the fusion pore compared to the overall size of the virus particles, and further characteristic size of late endosomes make it logistically difficult. However, such a possibility provides an intriguing possibility for future research, and can be falsified by an uncoating experiment with kinesin inhibitors.

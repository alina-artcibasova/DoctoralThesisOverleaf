The mostly cytoplasmic deacetylase HDAC6 plays an important role in the management of misfolded proteins and the stress response: it is a critical component of the aggresome pathway (Kawaguchi et al., 2003) and it participates in the formation of stress granules (Kwon et al., 2007; Saito et al., 2019). HDAC6 exerts its diverse biological functions by deacetylating various substrates such as tubulin, HSP90 or cortactin (Hubbert et al., 2002; Kovacs et al., 2005; Zhang et al., 2007; Zhang et al., 2003), and also by binding to unanchored ubiquitin chains via a conserved zinc finger domain (Hook et al., 2002; Seigneurin-Berny et al., 2001). During influenza A uncoating after membrane fusion in late endosomes, HDAC6 mediates physical connections between the virus and molecular motors such as dyneins and myosins (Figure 1A). To approach a mechanistic and quantitative understanding of HDAC6-mediated uncoating, we first asked whether it is physically plausible that molecular motors exert sufficient forces on the viral M1 capsid for its breakage. In a tug-of-war scenario, forces exerted by these motors could break the M1 capsid and release the viral ribonucleoprotein (RNP) complex (Figure 1A, B) (Banerjee et al., 2014).

To represent the underlying biophysics of a tug-of-war mechanism, we developed a mathematical model featuring the capsid at the stage when it is exposed via the fusion pore, interactions between the capsid and molecular motors, and interactions between motors and the host cell’s cytoskeleton (Figure 2A). Specifically, we assume that M1 proteins (the masses) are arranged in a regular mesh approximately of the size of the fusion pore (Hilsch et al., 2014); they are connected to each other by elastic bonds (springs with Morse potentials; see Methods for details). We represent the cytoskeleton by a single, randomly directed microtubule and by a denser network of actin filaments with a randomly located nucleation point. Molecular motors can be connected (directly or indirectly, which we do not distinguish in this model) to the M1 proteins exposed to the cytoplasm and to the cytoskeleton, and thereby exert forces. Specifically, dynein motors can walk along the microtubule in a single direction, while myosin motors can walk along actin filaments in random directions. We compute the resulting forces through a tug-of-war model with experimentally determined motor characteristics (Gennerich et al., 2007; Muller et al., 2008; Norstrom et al., 2010), which we modified to represent dyneins, kinesins, and positive- and negative-direction myosins. Importantly, the model considers that the force exerted by each individual motor depends on all the other motors bound to the same cargo. If these combined motor forces lead to the distance between any two neighboring M1 nodes exceeding the diameter of the viral ribonucleoprotein (RNP) complex for a sufficient duration, we classify the capsid as broken (see Methods for details).

Model-predicted capsid breakage probabilities for varying numbers and combinations of myosin and dynein motors are shown in Figure 2B. As expected, dyneins alone (zero myosins scenario) were unable to pull the capsid apart because all dynein motors exert their force in the same direction, constrained by the orientation of the microtubule adjacent to the fusion pore. The model predicted that myosins alone (zero dyneins scenario) could exert sufficient forces in different directions to uncoat the viral capsid. In this scenario, we achieved maximum ~50\% capsid breakage probability when 9 out of the inner 16 capsid nodes were occupied by myosins. However, with higher myosin occupancy, interference between myosin motors reduced this probability to approximately 30\%. Surprisingly, our simulations predict that the interaction between a single dynein motor and 5-7 myosin motors leads to 80-90\% probabilities of capsid breakage. Introducing more dyneins still allows high breakage, but requires a larger number of myosin motors.

To assess the robustness of these predictions, we analyzed the effects of the interaction strength between M1 proteins on capsid breakage probability. We varied the stiffness of the M1-M1 bond and the dissociation energy well depth for the Morse potential, which we originally inferred from indirect measurements and as approximations from other viruses (see Methods). Strengthening the M1-M1 bond led to lower breakage probabilities, and vice versa (Figure S1C). These changes, however, are primarily quantitative and do not affect the main conclusions: molecular motors, in realistic geometries and with realistic characteristics, can exert sufficient forces for virus uncoating, and the capsid breakage probability increases with the synergistic interaction of (few) dyneins and (primarily) myosins attached to the virus capsid.
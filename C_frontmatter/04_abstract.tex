%*******************************************************
% Abstract
%*******************************************************
%\renewcommand{\abstractname}{Abstract}
\pdfbookmark[1]{Abstract}{Abstract}
\begingroup
\let\clearpage\relax
\let\cleardoublepage\relax
\let\cleardoublepage\relax

\chapter*{Abstract}

Influenza viruses cause seasonal outbreaks leading to millions of cases of severe illness, and up to 650,000 deaths annually worldwide. A small number of antiviral drugs have been developed so far to reduce the influenza burden in vulnerable populations. Rapid mutations make antiviral drug development a challenge, since antivirals acting directly on viral proteins create an evolutionary pressure on the virus.

A novel drug development approach shifts the focus from viral proteins to host proteins which are hijacked to assist viral infection. One such protein, called HDAC6 is involved in influenza uncoating - a process in which viral protective layers are removed to release its genetic material. Uncoating is a dynamic, active and regulated process, in which host components, such as molecular motors, functionally interact with the virus. A one-dimensional tug-of-war mechanism of uncoating, in which microtubule motor proteins exert mechanical forces onto the viral protein capsid and initiate capsid breaking, has been suggested for several viruses. However, no mechanistic details or quantitative prediction of its feasibility are available.

To get a deep mechanistic understanding of HDAC6-mediated uncoating we combine available experimental knowledge with computational analysis to quantitatively elucidate mechanisms of influenza A virus (IAV) uncoating \textit{in vivo}.

First, we develop a biophysical model to demonstrate that interactions between capsid M1 proteins, host histone deacetylase 6 (HDAC6), and molecular motors myosin 10 and dynein could physically break the capsid in a two-dimensional tug-of-war mechanism.

Second, we develop a biochemical-biophysical model to identify the role of unanchored ubiquitin chains in HDAC6-mediated uncoating. We use this model to robustly predict uncoating efficiency \textit{in vivo} in normal and perturbed conditions.

Finally, we expand this biochemical-biophysical model and build up on existing kinetic models of influenza infection to predict the influence of DARPin-F10 - a novel drug-like compound inhibiting HDAC6-mediated uncoating - on influenza infection.

The study demonstrates how such layered mechanism-based mathematical models could help to answer specific questions about underlying processes of viral infections, and help formulate novel strategies for broad-range antiviral treatment.

\endgroup

\cleardoublepage%

\begingroup
\let\clearpage\relax
\let\cleardoublepage\relax
\let\cleardoublepage\relax

\begin{otherlanguage}{ngerman}
\pdfbookmark[1]{Zusammenfassung}{Zusammenfassung}
\chapter*{Zusammenfassung}

Deutsche Zusammenfassung hier.

\end{otherlanguage}

\endgroup

\vfill
%*******************************************************
% Abstract
%*******************************************************
%\renewcommand{\abstractname}{Abstract}
\pdfbookmark[1]{Abstract}{Abstract}
\begingroup
\let\clearpage\relax
\let\cleardoublepage\relax
\let\cleardoublepage\relax

\chapter*{Abstract}

Influenza viruses cause seasonal outbreaks leading to millions of cases of severe illness, and up to 650,000 deaths annually worldwide. A small number of antiviral drugs have been developed so far to reduce the influenza burden in vulnerable populations. Rapid mutations make antiviral drug development a challenge, since antivirals acting directly on viral proteins create an evolutionary pressure on the virus.

A novel drug development approach shifts the focus from viral proteins to host proteins which are hijacked to assist viral infection. One such protein, called HDAC6 is involved in influenza uncoating - a process in which viral protective layers are removed to release its genetic material. Uncoating is a dynamic, active and regulated process, in which host components, such as molecular motors, functionally interact with the virus. A one-dimensional tug-of-war mechanism of uncoating, in which microtubule motor proteins exert mechanical forces onto the viral protein capsid and initiate capsid breaking, has been suggested for several viruses. However, no mechanistic details or quantitative prediction of its feasibility are available.

To get a deep mechanistic understanding of HDAC6-mediated uncoating we combine available experimental knowledge with computational analysis to quantitatively elucidate mechanisms of influenza A virus (IAV) uncoating \textit{in vivo}.

First, we develop a biophysical model to demonstrate that interactions between capsid M1 proteins, host histone deacetylase 6 (HDAC6), and molecular motors myosin 10 and dynein could physically break the capsid in a two-dimensional tug-of-war mechanism.

Second, we develop a biochemical-biophysical model to identify the role of unanchored ubiquitin chains in HDAC6-mediated uncoating. We use this model to robustly predict uncoating efficiency \textit{in vivo} in normal and perturbed conditions.

Finally, we expand this biochemical-biophysical model and build up on existing kinetic models of influenza infection to predict the influence of DARPin-F10 - a novel drug-like compound inhibiting HDAC6-mediated uncoating - on influenza infection.

The study demonstrates how such layered mechanism-based mathematical models could help to answer specific questions about underlying processes of viral infections, and help formulate novel strategies for broad-range antiviral treatment.

\endgroup

\cleardoublepage%

\begingroup
\let\clearpage\relax
\let\cleardoublepage\relax
\let\cleardoublepage\relax

\begin{otherlanguage}{ngerman}
\pdfbookmark[1]{Zusammenfassung}{Zusammenfassung}
\chapter*{Zusammenfassung}


Influenzaviren verursachen saisonale Ausbrüche mit Millionen schweren Krankheitsfällen und weltweit bis zu 650.000 Todesfällen pro Jahr. Bisher wurden nur wenige antivirale Medikamente entwickelt, die die Influenza-Belastung in gefährdeten Bevölkerungsgruppen reduzieren. Schnelle Mutationen machen die Entwicklung antiviraler Medikamente zu einer Herausforderung, da diese direkt auf die Virus-Proteine wirken und somit einen evolutionären Druck auf das Virus ausüben.

In einem neuen Ansatz zur Medikamentenentwicklung wird der Fokus von Virus-Proteinen auf Wirtsproteine verlagert, die die Virusinfektion unterstützen. Eines dieser Proteine, genannt HDAC6, ist an der Virus-Entmantelung beteiligt – ein Prozess, in dem Virus-Schutzschichten (Kapsid) entfernt werden, um das genetische Material des Virus freizusetzen. Die Virus-Entmantelung ist ein dynamischer, aktiver und regulierter Prozess, in dem Wirtsmoleküle wie Motorproteine mit dem Virus interagieren. Für einige Viren wurde ein eindimensionaler Tauzieh-Mechanismus vorgeschlagen, bei dem Mikrotubuli-Motorproteine Zieh-Kräfte auf das Virus-Kapsid ausüben und es somit aufbrechen. Allerdings sind die mechanistischen Einzelheiten dieses Prozesses unbekannt und es gibt keine quantitative Vorhersagen zu dessen Wahrscheinlichkeit.

Mit dem Ziel die HDAC6-vermittelte Virus-Entmantelung bis ins Detail zu verstehen, kombinieren wir experimentelle Ergebnisse mit rechnergestützten Analysen, um Mechanismen der Influenza-A-Virus-Entmantelung \textit{in vivo} quantitativ zu erklären.  

Zunächst entwickeln wir ein biophysikalisches Modell, um zu zeigen, dass Wechselwirkungen zwischen Kapsid-M1-Proteinen, Histon-Deacetylase 6 (HDAC6) des Wirts und den molekularen Motoren Myosin 10 und Dynein das Kapsid in einem zweidimensionalen Tauzieh-Mechanismus aufbrechen können.  

Als Nächstes entwickeln wir ein biochemisch-biophysikalisches Modell, um die Rolle nicht verankerter Ubiquitin-Ketten in der HDAC6-vermittelten Entmantelung zu verstehen. Wir verwenden dieses Modell, um die Entmantelungs-Effizienz \textit{in vivo} unter normalen und modifizierten Bedingungen zuverlässig vorherzusagen.

Zuletzt erweitern wir das biochemisch-biophysikalische Modell und bauen auf kinetischen Influenza-Infektionsmodellen auf, um den Einfluss von DARPin-F10 – einer neuen arzneimittelähnlichen Verbindung, die die HDAC6-vermittelte Entmantelung hemmt – auf die Influenza-Infektion vorherzusagen. 

Diese Arbeit zeigt, wie mehrschichtige, mechanistische mathematische Modelle dazu beitragen können, spezifische Fragen über die zugrunde liegenden Prozesse von Virusinfektionen zu beantworten und neue Strategien für eine antivirale Therapie mit breitem Anwendungsbereich vorzuschlagen.

\end{otherlanguage}

\endgroup

\vfill
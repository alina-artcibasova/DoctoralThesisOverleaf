%*******************************************************
% Abstract
%*******************************************************
%\renewcommand{\abstractname}{Abstract}
\pdfbookmark[1]{Abstract}{Abstract}
\begingroup
\let\clearpage\relax
\let\cleardoublepage\relax
\let\cleardoublepage\relax

\chapter*{Abstract}

Influenza viruses cause seasonal outbreaks leading to millions of cases of severe illness, and up to 650,000 deaths annually worldwide. A small number of antiviral drugs have been developed so far to reduce the influenza burden in vulnerable populations. Rapid mutations make antiviral drug development a challenge, since antivirals acting directly on viral proteins create an evolutionary pressure on the virus.

A novel drug development approach shifts the focus from viral proteins to host proteins which are hijacked to assist viral infection. One such protein, called histone deacetylase 6 (HDAC6) is involved in influenza uncoating - a process in which viral protective layers are removed to release its genetic material. Uncoating is a dynamic, active and regulated process, in which host components, such as molecular motors, functionally interact with the virus. A one-dimensional tug-of-war mechanism of uncoating, in which microtubule motor proteins exert mechanical forces onto the viral protein capsid and initiate capsid breaking, has been suggested for several viruses. However, no mechanistic details or quantitative prediction of its feasibility are available.

In this work, to get a deep mechanistic understanding of HDAC6-mediated influenza A virus (IAV) uncoating \textit{in vivo} we combined available experimental knowledge with computational analysis.

First, we develop a biophysical model to demonstrate that interactions between capsid M1 proteins, host HDAC6, and molecular motors myosin 10 and dynein could physically break the capsid in a two-dimensional tug-of-war mechanism.

Second, we develop a biochemical-biophysical model to identify the role of unanchored ubiquitin chains in HDAC6-mediated uncoating. We use this model to robustly predict uncoating efficiency \textit{in vivo} in normal and perturbed conditions.

Finally, we expand this biochemical-biophysical model and build up on existing kinetic models of influenza infection to predict the influence of DARPin-F10 - a novel drug-like compound inhibiting HDAC6-mediated uncoating - on influenza infection.

The study demonstrates how such layered mechanism-based mathematical models could help to answer specific questions about underlying processes of viral infections, and help formulate novel strategies for broad-range antiviral treatment.

\endgroup

\cleardoublepage%

\begingroup
\let\clearpage\relax
\let\cleardoublepage\relax
\let\cleardoublepage\relax

\begin{otherlanguage}{ngerman}
\pdfbookmark[1]{Zusammenfassung}{Zusammenfassung}
\chapter*{Zusammenfassung}


Influenzaviren verursachen saisonale Ausbrüche mit Millionen schweren Krankheitsfällen und weltweit bis zu 650.000 Todesfällen pro Jahr. Bisher wurden nur wenige antivirale Medikamente entwickelt, die die Influenza-Belastung in gefährdeten Bevölkerungsgruppen reduzieren. Schnelle Mutationen machen die Entwicklung antiviraler Medikamente zu einer Herausforderung, da diese direkt auf die Virus-Proteine wirken und somit einen evolutionären Druck auf das Virus ausüben.

In einem neuen Ansatz zur Medikamentenentwicklung wird der Fokus von Virus-Proteinen auf Wirtsproteine verlagert, die die Virusinfektion unterstützen. Eines dieser Proteine, genannt HDAC6, ist an der Virus-Entmantelung beteiligt – ein Prozess, in dem Virus-Schutzschichten (Kapsid) entfernt werden, um das genetische Material des Virus freizusetzen. Die Virus-Entmantelung ist ein dynamischer, aktiver und regulierter Prozess, in dem Wirtsmoleküle wie Motorproteine mit dem Virus interagieren. Für einige Viren wurde ein eindimensionaler Tauzieh-Mechanismus vorgeschlagen, bei dem Mikrotubuli-Motorproteine Zieh-Kräfte auf das Virus-Kapsid ausüben und es somit aufbrechen. Allerdings sind die mechanistischen Einzelheiten dieses Prozesses unbekannt und es gibt keine quantitative Vorhersagen zu dessen Wahrscheinlichkeit.

Mit dem Ziel die HDAC6-vermittelte Virus-Entmantelung bis ins Detail zu verstehen, kombinieren wir experimentelle Ergebnisse mit rechnergestützten Analysen, um Mechanismen der Influenza-A-Virus-Entmantelung \textit{in vivo} quantitativ zu erklären.  

Zunächst entwickeln wir ein biophysikalisches Modell, um zu zeigen, dass Wechselwirkungen zwischen Kapsid-M1-Proteinen, Histon-Deacetylase 6 (HDAC6) des Wirts und den molekularen Motoren Myosin 10 und Dynein das Kapsid in einem zweidimensionalen Tauzieh-Mechanismus aufbrechen können.  

Als Nächstes entwickeln wir ein biochemisch-biophysikalisches Modell, um die Rolle nicht verankerter Ubiquitin-Ketten in der HDAC6-vermittelten Entmantelung zu verstehen. Wir verwenden dieses Modell, um die Entmantelungs-Effizienz \textit{in vivo} unter normalen und modifizierten Bedingungen zuverlässig vorherzusagen.

Zuletzt erweitern wir das biochemisch-biophysikalische Modell und bauen auf kinetischen Influenza-Infektionsmodellen auf, um den Einfluss von DARPin-F10 – einer neuen arzneimittelähnlichen Verbindung, die die HDAC6-vermittelte Entmantelung hemmt – auf die Influenza-Infektion vorherzusagen. 

Diese Arbeit zeigt, wie mehrschichtige, mechanistische mathematische Modelle dazu beitragen können, spezifische Fragen über die zugrunde liegenden Prozesse von Virusinfektionen zu beantworten und neue Strategien für eine antivirale Therapie mit breitem Anwendungsbereich vorzuschlagen.

\end{otherlanguage}

\endgroup

\cleardoublepage%

\begingroup
\let\clearpage\relax
\let\cleardoublepage\relax
\let\cleardoublepage\relax

\begin{otherlanguage}{russian}
\pdfbookmark[1]{РЕЗЮМЕ}{РЕЗЮМЕ}
\chapter*{РЕЗЮМЕ}


Вирусы гриппа являются причиной миллионов серьезных случаев заболевания, и вплоть до 650,000 смертей ежегодно по всему миру. Немногочисленные антивирусные препараты против гриппа используются, чтобы облегчить протекание болезни в уязвимых группах населения. Высокая частота мутаций вируса усложняет разработку новых антивирусных препаратов, так как препараты, действующие непосредственно на вирусные белки, оказывают эволюционное давление на вирус.

Новый подход к разработке антивирусных препаратов фокусируется не на вирусных белках, а на белках клеток хозяина, которые вовлечены в процесс вирусной инфекции. Один из таких белков - гистоновая деацетилаза 6 (HDAC6). HDAC6 участвует в стадии обнажения вируса гриппа - процессe, в котором защитные слои вируса удаляются, освобождая его генетический материал. Обнажение вируса - это быстрый, активный и регулируемый процесс, в котором компоненты клеток хозяина функционально взаимодействуют с вирусом. Чтобы объяснить стадию обнажения многих вирусов, ранее был предложен одномерный механизм “перетягивания каната” (“tug-of-war”), в котором механические силы клеточных молекулярных моторов микротрубочек действуют на капсид вируса и инициируют его поломку. Однако, детальный механизм этого процесса, так же как и количественные предсказания его осуществимости на данный момент неизвестны.

В данном исследовании мы объединили доступные экспериментальные знания и вычислительный анализ, чтобы количественно описать механизм HDAC6-ассистируемой стадии обнажения вируса гриппа \textit{in vivo}.

Во-первых, мы разработали биофизичекую модель, демонстрирующую, что механические взаимодействия между белком вирусного капсида М1, белками клеток хозяина - HDAC6 и молекулярными моторами миозином 10 и динеином - способны физически сломать капсид в процессе двухмерного “перетягивания каната”.

Во-вторых, мы разработали биохимическо-биофизическую модель, чтобы определить роль незакрепленных цепей убиквитина в HDAC6-ассистируемой стадии обнажения вируса гриппа. Мы использовали эту модель, чтобы предсказать эффективность вирусного обнажения в нормальном и нарушенных клеточных состояниях.

Наконец, мы расширили эту биохимическо-биофизическую модель и модифицировали существующие кинетические модели инфекции гриппа. Мы использовали эту новую модифицированную модель, чтобы предсказать протекание вирусной инфекции под влиянием нового потенциального лекарственного соединения DARPin-F10, ингибирующего HDAC6-ассистируемую стадию обнажения вируса гриппа.

Наше исследование демонстрирует, что такие многослойные математические модели, детально описывающие механизм биологического процесса,  позволяют ответить на специфические вопросы об основополагающих процессах в вирусных инфекциях, и могут помочь в разработке новых антивирусных лекарственных препаратов.

\end{otherlanguage}


\endgroup

\vfill